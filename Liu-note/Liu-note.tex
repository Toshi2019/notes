% -----------------------
% preamble
% -----------------------
% Don't change preamble code yourself. If you add something(usepackage, newtheorem, newcommand, renewcommand),
% please tell them editor of institutional paper of RUMS.

%documentclass
%------------------------
\documentclass[12pt, a4paper, dvipdfmx]{jsarticle}


%usepackage
%------------------------
\usepackage{amsmath}
\usepackage{amsthm}
%\usepackage[psamsfonts]{amssymb}
\usepackage{color}
\usepackage{ascmac}
\usepackage{amsfonts}
\usepackage{mathrsfs}
\usepackage{mathtools}
\usepackage{amssymb}
\usepackage{graphicx}
\usepackage{fancybox}
\usepackage{enumerate}
\usepackage{verbatim}
\usepackage{subfigure}
\usepackage{proof}
\usepackage{listings}
\usepackage{otf}
\usepackage{algorithm}
\usepackage{algorithmic}
\usepackage{tikz}
\usepackage[all]{xy}
\usepackage{amscd}

\usepackage{pb-diagram}

\usepackage[dvipdfmx]{hyperref}
\usepackage{xcolor}
\definecolor{darkgreen}{rgb}{0,0.45,0} 
\definecolor{darkred}{rgb}{0.75,0,0}
\definecolor{darkblue}{rgb}{0,0,0.6} 
\hypersetup{
    colorlinks=true,
    citecolor=darkgreen,
    linkcolor=darkred,
    urlcolor=darkblue,
}
\usepackage{pxjahyper}

\usepackage{enumitem}

\usepackage{bbm}

\usetikzlibrary{cd}
\def\objectstyle{\displaystyle}

%theoremstyle
%--------------------------
\theoremstyle{definition}


%newtheoem
%--------------------------
%If you want to use theorem environment in Japanece, You can use these code. 
%Attention
%--------------------------
%all theorem enivironment number depend on only section number.
\newtheorem{Axiom}{公理}[section]
\newtheorem{Definition}[Axiom]{定義}
\newtheorem{Theorem}[Axiom]{定理}
\newtheorem{Proposition}[Axiom]{命題}
\newtheorem{Lemma}[Axiom]{補題}
\newtheorem{Corollary}[Axiom]{系}
\newtheorem{Example}[Axiom]{例}
\newtheorem{Claim}[Axiom]{主張}
\newtheorem{Property}[Axiom]{性質}
\newtheorem{Remark}[Axiom]{注意}
\newtheorem{Question}[Axiom]{問}
\newtheorem{Problem}[Axiom]{問題}
\newtheorem{Consideration}[Axiom]{考察}
\newtheorem{Alert}[Axiom]{警告}
\newtheorem{Notation}[Axiom]{記号}



%----------------------------
%If you want to use theorem environment with no number in Japanese, You can use these code.
\newtheorem*{Axiom*}{公理}
\newtheorem*{Definition*}{定義}
\newtheorem*{Theorem*}{定理}
\newtheorem*{Proposition*}{命題}
\newtheorem*{Lemma*}{補題}
\newtheorem*{Example*}{例}
\newtheorem*{Corollary*}{系}
\newtheorem*{Claim*}{主張}
\newtheorem*{Property*}{性質}
\newtheorem*{Remark*}{注意}
\newtheorem*{Question*}{問}
\newtheorem*{Problem*}{問題}
\newtheorem*{Consideration*}{考察}
\newtheorem*{Alert*}{警告}

%--------------------------
%If you want to use theorem environment in English, You can use these code.
%--------------------------
%all theorem enivironment number depend on only section number.
\newtheorem{Axiom+}{Axiom}[section]
\newtheorem{Definition+}[Axiom+]{Definition}
\newtheorem{Theorem+}[Axiom+]{Theorem}
\newtheorem{Proposition+}[Axiom+]{Proposition}
\newtheorem{Lemma+}[Axiom+]{Lemma}
\newtheorem{Example+}[Axiom+]{Example}
\newtheorem{Corollary+}[Axiom+]{Corollary}
\newtheorem{Claim+}[Axiom+]{Claim}
\newtheorem{Property+}[Axiom+]{Property}
\newtheorem{Remark+}[Axiom+]{Attention}
\newtheorem{Question+}[Axiom+]{Question}
\newtheorem{Problem+}[Axiom+]{Problem}
\newtheorem{Consideration+}[Axiom+]{Consideration}
\newtheorem{Alert+}{Alert}

%commmand
%----------------------------
\newcommand{\N}{\mathbb{N}}
\newcommand{\Z}{\mathbb{Z}}
\newcommand{\Q}{\mathbb{Q}}
\newcommand{\R}{\mathbb{R}}
\newcommand{\C}{\mathbb{C}}
\newcommand{\F}{\mathcal{F}}
\newcommand{\X}{\mathcal{X}}
\newcommand{\Y}{\mathcal{Y}}
\newcommand{\Hil}{\mathcal{H}}
\newcommand{\RKHS}{\Hil_{k}}
\newcommand{\Loss}{\mathcal{L}_{D}}
\newcommand{\MLsp}{(\X, \Y, D, \Hil, \Loss)}
\newcommand{\p}{\partial}
\newcommand{\h}{\mathscr}
\newcommand{\mcal}{\mathcal}
\newcommand{\lan}{\langle}
\newcommand{\ran}{\rangle}
\newcommand{\pal}{\parallel}
\newcommand{\dip}{\displaystyle}
\newcommand{\e}{\varepsilon}
\newcommand{\dl}{\delta}
\newcommand{\pphi}{\varphi}
\newcommand{\ti}{\tilde}

\newcommand{\inc}{\hookrightarrow}
\newcommand{\cni}{\hookleftarrow}


\renewcommand{\P}{\mathbb{P}}
\newcommand{\Probsp}{(\Omega, \F, \P)}

%new definition macro
%-------------------------
\def\inner<#1>{\langle #1 \rangle}

\usepackage[stable]{footmisc}
%\renewcommand{\proofname}{\textrm{証明}} % 「証明」の見出しを日本語にする
\newcommand{\pr}{\mathop{\mathrm{pr}}\nolimits} % 射影の記号を斜字体にしない
\newcommand{\id}{\mathop{\mathrm{id}}\nolimits} % 恒等写像
\newcommand{\Ob}{\mathop{\mathrm{Ob}}\nolimits}
\numberwithin{equation}{section} % 式番号を「(3.5)」のように印刷
\newcommand{\Hom}{\mathop{\mathrm{Hom}}\nolimits}
\newcommand{\Mod}{\mathop{\mathrm{Mod}}\nolimits}
\newcommand{\End}{\mathop{\mathrm{End}}\nolimits}
\newcommand{\Aut}{\mathop{\mathrm{Aut}}\nolimits}
\newcommand{\Mor}{\mathop{\mathrm{Mor}}\nolimits}

% 圏の記号など
\newcommand{\Set}{{\bf Set}}
\newcommand{\Vect}{{\bf Vect}}
\newcommand{\FDVect}{{\bf FDVect}}
\newcommand{\Ring}{{\bf Ring}}
\newcommand{\Ab}{{\bf Ab}}
\newcommand{\CGA}{{\bf CGA}}
\newcommand{\GVect}{{\bf GVect}}
\newcommand{\Lie}{{\bf Lie}}
\newcommand{\dLie}{{\bf Liec}}
\newcommand{\Op}{\mathop{\mathrm{Open}}\nolimits}
\newcommand{\op}{{\rm op}} % 双対を取る記号



%\usepackage{eufrak}
\def\rnum#1{\expandafter{\romannumeral #1}} 
\def\Rnum#1{\uppercase\expandafter{\romannumeral #1}} 

\newcommand{\bm}[1]{{\mbox{\boldmath $#1$}}}

\newcommand{\gG}{\mathfrak{g}}
\newcommand{\gh}{\mathfrak{h}}
\newcommand{\lL}{\mathop{\mathrm{L}}\nolimits}

\newcommand{\cat}[1]{\textup{\textsf{#1}}}% for categories
\newcommand{\fun}[1]{\textup{#1}}%for functors


%for yoneda
\font\maljapanese=dmjhira at 2ex % you can change this "2ex" value
\def\yo{\textrm{\maljapanese\char"48}}

%
% \qed を自動で入れない proof 環境を再定義
%
\makeatletter
\renewenvironment{proof}[1][\proofname]{\par
  \normalfont
  \topsep6\p@\@plus6\p@ \trivlist
  \item[\hskip\labelsep{\textbf{#1}}\@addpunct{\textbf{. }}]\ignorespaces
}{%
  \endtrivlist
}
\renewcommand{\proofname}{証明}
\makeatother




%----------------------------
%documenet 
%----------------------------
% Your main code must be written between begin document and end document.
\title{Liu ゼミノート}
\author{}
\date{}
\begin{document}

\maketitle

\begin{abstract}
    2021 年春セメスターに行なった \cite{Liu} ゼミのノート. 
\end{abstract}

\section*{実施日}
\begin{itemize}
    \item 5/15: 定義\ref{def-psh}   -- 注意\ref{Rem-Bsh}
    \item 5/16: 補題\ref{lem-shCmp} -- 定義\ref{def-morPsh}
    \item 5/21: 定義\ref{def-stalk} -- 定義\ref{def-shfify}
\end{itemize}

\setcounter{section}{1}

\section{Ringed topological spaces}

\subsection{Sheaves}

\begin{Definition}\label{def-psh}
    $X$ を位相空間とする. 
    $X$上の \textbf{(アーベル群の) 前層} (presheaf) 
    $\F$は次のデータからなる. 

    -- $X$ の各開部分集合 $U$ に対するアーベル群 $\F(U)$, そして

    -- 部分開集合の各組 $V \subset U$ に対する
    群準同型 (\textbf{制限写像}) $\rho_{UV} \colon \F(U)\to \F(V)$
    で, 次の条件を満たすもの: 
    \begin{enumerate}
        \renewcommand{\labelenumi}{({\arabic{enumi}})}
        \item $\F(\varnothing) = 0$; 
        \item $\rho_{UU} = \id$; 
        \item $W \subset V \subset U$ ならば, 
        $\rho_{UW} = \rho_{VW} \circ \rho_{UV}$.
    \end{enumerate}
    元$s\in \F(U)$を$\F$の$U$上の\textbf{切断} (section) という. 
    $s|_V$ で $\rho_{UV}(s)\in\F(V)$ を表し, 
    $s$ の $V$ への\textbf{制限} (restriction) とよぶ. 
\end{Definition}

\begin{Definition}\label{def-shv}
    前層 $\F$ が次の条件をみたすとき, \textbf{層} であるという. 
    \begin{enumerate}
        \renewcommand{\labelenumi}{({\arabic{enumi}})}
        \setcounter{enumi}{3}
        \item (\textbf{一意性})\label{shv-uniq}
        $U$を$X$の開部分集合とし, 
        $s\in \F(U)$, $\{U_i\}_i$を
        開部分集合$U_i$による$U$ の被覆とする. 
        全ての$i$に対し$s|_{U_i} = 0$ ならば, $s = 0$
        である. 
        \item (\textbf{局所切断の貼り合わせ})\label{shv-glu}
        (\ref{shv-uniq}) の記法を用いる. 
        $s_i\in \F(U_i)$, $i\in I$
        を切断で
        $s_i|_{U_i\cap U_j} = s_j|_{U_i\cap U_j}$
        をみたすものとする. 
        このとき, 切断$s\in \F(U)$で
        $s|_{U_i} = s_i$
        をみたすものが存在する 
        (この切断$s$は (\ref{shv-uniq}) より一意である). 
    \end{enumerate}
\end{Definition}

\begin{proof}[定義\ref{def-shv}は何を言っているのか] 
    次の射を考える. 
    \begin{align*}
        \prod_{i\in I}\rho_{UU_i} \colon 
        \F(U)\to \prod_{i\in I}\F(U_i);
        \quad 
        s\mapsto (s|_{U_i})_{i\in I}.
    \end{align*}

    (\ref{shv-uniq}) は
    $\prod_{i\in I} \rho_{UU_i}$が単射である, と言っている. 
    
    (\ref{shv-uniq})について. 
    $\prod_{i\in I}\F(U_i)$の部分群
    \begin{align*}
        M \coloneqq \left\{\left.
            (s_i)_{i\in I} \in \prod_{i\in I}\F(U_i) 
            \right|\ \forall i,j\in I \ 
            s_i|_{U_i\cap U_j} = s_j|_{U_i\cap U_j} \right\}
    \end{align*}
    を考える. (\ref{shv-glu}) は
    $\prod_{i\in I}\rho_{UU_i}\colon \F(U)\to M$
    が全射であるということである. 

    つまり, (\ref{shv-uniq}), (\ref{shv-glu}) を合わせると, 
    $\prod_{i\in I}\rho_{UU_i}$ によって 
    $\F(U)$と$M$が同形になるといっている.
\end{proof}

\begin{proof}[定義\ref{def-shv} (\ref{shv-glu}) の一意性の証明] 
    切断$s, s' \in \F(U)$が
    $s|_{U_i} = s_i$, $s'|_{U_i} = s_i$
    をみたすとする. このとき, 
    \begin{align*}
        (s-s')|_{U_i} 
        &= \rho_{UU_i}(s-s') \\
        &= \rho_{UU_i}(s) - \rho_{UU_i}(s')\\
        &= s|_{U_i} - s'|_{U_i} = s_i - s_i = 0. 
    \end{align*}
    したがって, 定義\ref{def-shv} (\ref{shv-uniq}) より$s-s' = 0$. 
    すなわち. $s = s'$.
\end{proof}

同様にして, \textbf{環上の層}, 
固定した環の上の\textbf{代数上の層}等々を定義できる. 
$\F$の\textbf{部分層}$\F'$も自然な概念である: 
$\F'(U)$は$\F(U)$の部分群であり, 
制限$\rho'_{UV}$は$\rho_{UV}$によって引き起こされる. 

\begin{proof}[部分層の定義] 
    まず部分関手の定義を復習する. 
    $F,G\colon \mcal{C} \to \mcal{C}'$
    を関手とし, $\pphi \colon F\to G$ を関手の射とする. 
    $\mcal{C}$の全ての対象 $X$ に対し, $F(X)\subset G(X)$
    であり$\pphi(X)$が包含$ F(X)\inc G(X)$となるとき, 
    $F$を$G$の部分関手という. 

    $G\colon \mcal{C}\to\mcal{C}'$を関手とし, 
    各対象$X\in\mcal{C}$に対し部分対象$F(X)\subset G(X)$
    が与えられているとき, $G$の部分関手$F$が定まるための条件は, 
    $\mcal{C}$の任意の射$f\colon X\to Y$に対し
    $G(f)(F(X))\subset F(Y)$となることである. 
    \[\begin{tikzcd}
        G(X) 
        \arrow[r, "G(f)"] 
        &
        G(Y)\\
        F(X)
        \arrow[r,hook, "G(f)"] 
        \arrow[u,hook] 
        &
        G(Y)\arrow[u,hook].
    \end{tikzcd}\]

    同様の条件をみたす関手
    $\mcal{G}\subset\F\colon \cat{Open}_X^{\op}\to\cat{Ab}$
    で, 層の条件 (\ref{shv-uniq}), (\ref{shv-glu}) 
    をみたすものとして$\F$の部分層$\mcal{G}$を定める. 
\end{proof}

\begin{Example}
    $X$を位相空間とする. $X$の任意の開集合$U$に対し, 
    $\mcal{C}(U) = \mathrm{C}^0(U,\R)$を
    $U$から$\R$への連続関数の集合とする. 
    制限$\rho_{UV}$は普通の関数の制限である. 
    このとき, $\mcal{C}$は$X$の層である.
    $\F(U) = \R^U$ を$U$上の$\R$に値をとる関数の集合とすると, 
    これは$\mcal{C}$を部分層としてもつ層$\F$を定める.  
\end{Example}

\begin{proof}
    $\mcal{C}(\varnothing) = 0$であるか: 
    空集合からの写像は包含$i\colon \varnothing\inc \R$のみであるから, 
    $\mcal{C}(\varnothing) = \{i\} \cong 0$である. 

    $\mcal{C}$が層になるための条件 
    (\ref{shv-uniq}), (\ref{shv-glu}) 
    をみたすことを示す. 
    $U = \cup_{i\in I} U_i$とする. 
    このとき, $s\colon U \to \R; s(x) = (x\in U_i)$を考える. 
    任意の$x\in U$に対し$x\in U_i$となる$i$が存在するので
    $s(x) = 0$である. 
    したがって, $\mcal{C}$は (\ref{shv-uniq}) をみたす. 

    $(s_i)_{i\in I}$, 
    $s_i|_{U_i\cap U_j} = s_j|_{U_i\cap U_j} 
    \ (i,j \in I)$
    とする. 写像$s\colon U \to \R$を
    $s(x) = s_i(x)\ (x\in U_i)$
    で定めたい. 
    いま, $x\in U$について$x\in U_i$をとなる$i\in I$
    が存在することは保証されている. 
    $x$が$U_i$の元であり, かつ$U_j$の元でもあるとする. 
    このとき$s_i(x) = s_j(x)$が成り立つことを示す. 
    \begin{align*}
        s_i(x) = s_i|_{U_i\cap U_j}(x), 
        s_j(x) = s_j|_{U_j\cap U_i}(x). 
    \end{align*} 
    したがって, $s_i(x) = s_j(x)$が成り立つので, 
    $s$は well-defined であり 
    (\ref{shv-glu}) が成り立つ. 

    $s\colon U \to \R$が連続写像であることを示す. 
    $x\in U$とし, $\e>0$を実数とする. 
    いま$x$が$U_i$の点であるとすると, $s_i$は連続なので
    開近傍$V_i\subset U_i$が存在する. 
    この$V_i$に対し, $V_i\subset U$である. 
    $x_0$を$V_i$の点とすると, 
    \begin{align*}
        | s(x)-s(x_0) | = | s_i(x)-s_i(x_0) | < \e
    \end{align*}
    が成り立つ. したがって, $s$は連続である. \qed
\end{proof}

\begin{proof}[関手$\F$について]
    $V\subset U$を$X$の開集合とする. 
    \begin{equation*}
        \vcenter{\xymatrix@C=36pt@R=36pt{
        \mcal{C}(U) \ar[r]^-{\rho^\mcal{C}_{UV}} \ar@{^{(}->}[d] 
        & 
        \mcal{C}(V) \ar@{^{(}->}[d] \\
        \F(U) \ar[r]^-{\rho^\F_{UV}}  
        & 
        \F(V)\ar@{}[lu]|{\circlearrowright}
        }}
    \end{equation*}
    
    $f\in\mcal{C}(U)$を切断とする. 
    $\rho^\mcal{C}_{UV}(f)=f|_V$であり, 
    $f|_V\in \mcal{C}(V)$が成り立つ. 
    よって, $\mcal{C}$は$\F$の部分層である. \qed
\end{proof}

\begin{Example} \label{ex-Ax}
    $A$を非自明なアーベル群とする. $X$を位相空間とする. 
    $\mcal{A}_X(U) = A$とし, $U$と$V$が空でなければ
    $\rho_{UV}=\id_A$とする. 
    これは$X$上の前層を定める. 
    一般には, $\mcal{A}_X$は層にはならない. 
    例えば, $X$が空でない2つの開集合の非交和だとすると, 
    層の条件 (\ref{shv-glu}) が成り立たない. 
\end{Example}

\begin{proof}
    $X = U \sqcup V (U,V\neq\varnothing)$とおき, 
    $a\neq b\in A$とする. 
    $a|_{U\cap V} = b|_{U\cap V}$
    であるが, $U\cup V$について, 
    $x\in A$で, $x|_U = a, x|_V = b$
    をみたすものが存在し, $x|_U = x$となるが, 
    これは$a\neq b$にムジュン. 
    したがって, $\mcal{A}_X$は
    層の条件 (\ref{shv-glu}) を満たさず, 
    層にはならない. \qed
\end{proof}

\begin{Remark} \label{Rem-rst}
    $U$が$X$の開集合であるとき, $X$上の任意の前層$\F$は
    自明な方法で, すなわち, $U$の任意の開部分集合$V$に対し, 
    $\F|_{U}(V)=\F(V)$とおくことで, 
    $U$上の前層$\F|_{U}$を引き起こす. 
    $\F$の$U$への\textbf{制限}(restriction) という. 
    $\F$が層になるならば, $\F|_{U}$もそうなる. 
\end{Remark}

\begin{proof}[コメント]
    $X$:top.\ sp., $V\subset U\subset X$: open のとき, 
    $\F\colon \cat{Open}X^\op\to \cat{Ab}$に対し
    $\F|_U\colon \cat{Open}U^\op\to \cat{Ab}$ : $V\mapsto \F(V)$
    としてとるということ. 
\end{proof}

\begin{Remark} \label{Rem-Bsh}
    $\mcal{B}$を$X$の部分集合の基底とする 
    (即ち, $\mcal{B}$は$X$の開集合のなす集合であって, 
    $X$の任意の開集合は$\mcal{B}$に属する部分開集合の合併であり, 
    有限の共通部分について安定であるということである). 
    $\mcal{B}$前層と$\mcal{B}$層を, 上の定義において 
    「$X$の開集合$U$」を「$\mcal{B}$に属する開集合$U$」で
    置き換えることで定められる. 
    このとき任意の$\mcal{B}$層$\F_0$は$X$上の層$\F$に
    一意に (より正確には同型を除いて一意に, 
    定義\ref{def-morPsh}参照) 拡張される. 
\end{Remark}

$\mcal{U} = \{U_i\}_i$を$X$の開集合族とする. 
$U = \cup_i U_i$とし$U_{ij} = U_i \cap U_j$とする. 
$X$上の任意の前層$\F$に対し, 
アーベル群の複体$C^\bullet(\mcal{U},\F)$
\begin{align*}
    0 \rightarrow 
    \F(Y) \xrightarrow{d_0} 
    \prod_i \F(U_i) \xrightarrow{d_1} 
    \prod_{i,j} \F(U_{ij})
\end{align*}
で, $d_0\colon s \mapsto (s|_{U_i})_i$と
$d_1\colon (s_i)_i \mapsto(s_i|_{U_{ij}} - s_j|_{U_{ij}})_{i,j}$
で定まるものを得る. 

\begin{Lemma} \label{lem-shCmp}
    上の記号のもとで, $\F$が層であることと,
    $X$の任意の開集合族$\mcal{U}$に対し
    $C^\bullet(\mcal{U},\F)$が完全であることは同値である. 
\end{Lemma}

\begin{proof}
    $\mcal{U}$
\end{proof}

\begin{Definition} \label{def-stalk}
    $\F$を$X$上の前層とし, $x\in X$とする. 
    $\F$の$x$における\textbf{茎} (stalk) とは, 
    群
    \begin{align*}
        \F_x = \varinjlim_{U\in I_x} \F(U)
    \end{align*}
    のことをいう. 
\end{Definition}

\begin{proof}[$I_x$について]
    $I_x\subset \cat{Open}X$を$x$の開近傍全体の成す順序集合で, 
    $U, V \in I_x$の順序を
    $U < V \Longleftrightarrow U \supset V$
    で定めたものとする ($I_x$を圏とみなしたとき, 
    $I_x$は$\cat{Open}X^\op$の充満部分圏になっている).     
\end{proof}
% \varinjlim  は帰納極限 (逆極限, 極限), 
% \varprojlim は射影極限 (順極限, 余極限)

\begin{Lemma}
    $s_x = t_x$
\end{Lemma}

\begin{Definition} \label{def-morPsh}
    $\alpha(U)$
\end{Definition}

\begin{Example}
    $\C \setminus \{0\}$
\end{Example}

\begin{Proposition}
    $\alpha_x$
\end{Proposition}

\begin{Corollary}
    $\alpha\colon \F \to \mcal{G}$
\end{Corollary}

\begin{Definition} \label{def-shfify}
    $\F^\dagger$
\end{Definition}

%=======================================
% 複体テンプレート
\begin{comment}
    \begin{equation*}
        \vcenter{\xymatrix@C=36pt{
        0 \ar[r] 
        & 
        \F(Y) \ar[r]^-{d_0} 
        & 
        \prod_i \F(U_i) \ar[r]^-{d_1} 
        & \prod_{i,j} \F(U_{ij}),
        }}
    \end{equation*}
\end{comment}


% 随伴テンプレート
\begin{comment}
\begin{equation}
    \begin{tikzcd}
    \cat{C}
    \arrow[r, bend left, "F"] 
    \arrow[r, phantom, "\perp"] 
    & 
    \cat{C}' \arrow[l, bend left, "G"]
    \end{tikzcd}
\end{equation}
\end{comment}




\begin{thebibliography}{20}
\par
  \bibitem[Liu]{Liu} Qing Liu, 
  \textit{Algebraic Geometry and Arithmetic Curves}, 
  Oxford Graduate Text in Mathematics, 
  \textbf{6}, 2010.
\end{thebibliography}
\end{document}