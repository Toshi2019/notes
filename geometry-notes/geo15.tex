
%Don't forget to delete
%showkeys
%overfullrule
%\date \ber \er \cmt



%\documentclass[12pt,leqno]{amsart}
\documentclass[12pt, a4paper, dvipdfmx, draft]{jsarticle}


% ------------------------
% usepackage
% ------------------------
\usepackage{algorithm}
\usepackage{algorithmic}
\usepackage{amscd}
\usepackage{amsfonts}
\usepackage{amsmath}
\usepackage[psamsfonts]{amssymb}
\usepackage{amsthm}
\usepackage{ascmac}
\usepackage{color}
\usepackage{enumerate}
\usepackage{fancybox}
\usepackage[stable]{footmisc}
\usepackage{graphicx}
\usepackage{listings}
\usepackage{mathrsfs}
\usepackage{mathtools}
\usepackage{otf}
\usepackage{pifont}
\usepackage{proof}
\usepackage{subfigure}
\usepackage{tikz}
\usepackage{verbatim}
\usepackage[all]{xy}

\usetikzlibrary{cd}



% ================================
% パッケージを追加する場合のスペース 
\usepackage{latexsym}
\usepackage{wrapfig}
\usepackage{layout}
\usepackage{url}

\usepackage{okumacro}
%=================================


% --------------------------
% theoremstyle
% --------------------------
\theoremstyle{definition}


% --------------------------
% newtheoem
% --------------------------

% 日本語で定理, 命題, 証明などを番号付きで用いるためのコマンドです. 
% If you want to use theorem environment in Japanece, 
% you can use these code. 
% Attention!
% All theorem enivironment numbers depend on 
% only section numbers.
\newtheorem{Axiom}{公理}[section]
\newtheorem{Definition}[Axiom]{定義}
\newtheorem{Theorem}[Axiom]{定理}
\newtheorem{Proposition}[Axiom]{命題}
\newtheorem{Lemma}[Axiom]{補題}
\newtheorem{Corollary}[Axiom]{系}
\newtheorem{Example}[Axiom]{例}
\newtheorem{Claim}[Axiom]{主張}
\newtheorem{Property}[Axiom]{性質}
\newtheorem{Attention}[Axiom]{注意}
\newtheorem{Question}[Axiom]{問}
\newtheorem{Problem}[Axiom]{問題}
\newtheorem{Consideration}[Axiom]{考察}
\newtheorem{Alert}[Axiom]{警告}
\newtheorem{Fact}[Axiom]{事実}


% 日本語で定理, 命題, 証明などを番号なしで用いるためのコマンドです. 
% If you want to use theorem environment with no number in Japanese, You can use these code.
\newtheorem*{Axiom*}{公理}
\newtheorem*{Definition*}{定義}
\newtheorem*{Theorem*}{定理}
\newtheorem*{Proposition*}{命題}
\newtheorem*{Lemma*}{補題}
\newtheorem*{Example*}{例}
\newtheorem*{Corollary*}{系}
\newtheorem*{Claim*}{主張}
\newtheorem*{Property*}{性質}
\newtheorem*{Attention*}{注意}
\newtheorem*{Question*}{問}
\newtheorem*{Problem*}{問題}
\newtheorem*{Consideration*}{考察}
\newtheorem*{Alert*}{警告}
\newtheorem*{Fact*}{事実}


% 英語で定理, 命題, 証明などを番号付きで用いるためのコマンドです. 
% If you want to use theorem environment in English, You can use these code.
%all theorem enivironment number depend on only section number.
\newtheorem{Axiom+}{Axiom}[section]
\newtheorem{Definition+}[Axiom+]{Definition}
\newtheorem{Theorem+}[Axiom+]{Theorem}
\newtheorem{Proposition+}[Axiom+]{Proposition}
\newtheorem{Lemma+}[Axiom+]{Lemma}
\newtheorem{Example+}[Axiom+]{Example}
\newtheorem{Corollary+}[Axiom+]{Corollary}
\newtheorem{Claim+}[Axiom+]{Claim}
\newtheorem{Property+}[Axiom+]{Property}
\newtheorem{Attention+}[Axiom+]{Attention}
\newtheorem{Question+}[Axiom+]{Question}
\newtheorem{Problem+}[Axiom+]{Problem}
\newtheorem{Consideration+}[Axiom+]{Consideration}
\newtheorem{Alert+}{Alert}
\newtheorem{Fact+}[Axiom+]{Fact}
\newtheorem{Remark+}[Axiom+]{Remark}

% ----------------------------
% commmand
% ----------------------------
% 執筆に便利なコマンド集です. 
% コマンドを追加する場合は下のスペースへ. 

% 集合の記号 (黒板文字)
\newcommand{\NN}{\mathbb{N}}
\newcommand{\ZZ}{\mathbb{Z}}
\newcommand{\QQ}{\mathbb{Q}}
\newcommand{\RR}{\mathbb{R}}
\newcommand{\CC}{\mathbb{C}}
\newcommand{\PP}{\mathbb{P}}
\newcommand{\KK}{\mathbb{K}}


% 集合の記号 (太文字)
\newcommand{\nn}{\mathbf{N}}
\newcommand{\zz}{\mathbf{Z}}
\newcommand{\qq}{\mathbf{Q}}
\newcommand{\rr}{\mathbf{R}}
\newcommand{\cc}{\mathbf{C}}
\newcommand{\pp}{\mathbf{P}}
\newcommand{\kk}{\mathbf{K}}

% 特殊な写像の記号
\newcommand{\ev}{\mathop{\mathrm{ev}}\nolimits} % 値写像
\newcommand{\pr}{\mathop{\mathrm{pr}}\nolimits} % 射影

% スクリプト体にするコマンド
%   例えば {\mcal C} のように用いる
\newcommand{\mcal}{\mathcal}

% 花文字にするコマンド 
%   例えば {\h C} のように用いる
\newcommand{\h}{\mathscr}

% ヒルベルト空間などの記号
\newcommand{\F}{\mcal{F}}
\newcommand{\X}{\mcal{X}}
\newcommand{\Y}{\mcal{Y}}
\newcommand{\Hil}{\mcal{H}}
\newcommand{\RKHS}{\Hil_{k}}
\newcommand{\Loss}{\mcal{L}_{D}}
\newcommand{\MLsp}{(\X, \Y, D, \Hil, \Loss)}

% 偏微分作用素の記号
\newcommand{\p}{\partial}

% 角カッコの記号 (内積は下にマクロがあります)
\newcommand{\lan}{\langle}
\newcommand{\ran}{\rangle}



% 圏の記号など
\newcommand{\Set}{{\bf Set}}
\newcommand{\Vect}{{\bf Vect}}
\newcommand{\FDVect}{{\bf FDVect}}
\newcommand{\Ring}{{\bf Ring}}
\newcommand{\Ab}{{\bf Ab}}
\newcommand{\Mod}{\mathop{\mathrm{Mod}}\nolimits}
\newcommand{\CGA}{{\bf CGA}}
\newcommand{\GVect}{{\bf GVect}}
\newcommand{\Lie}{{\bf Lie}}
\newcommand{\dLie}{{\bf Liec}}



% 射の集合など
\newcommand{\Map}{\mathop{\mathrm{Map}}\nolimits}
\newcommand{\Hom}{\mathop{\mathrm{Hom}}\nolimits}
\newcommand{\End}{\mathop{\mathrm{End}}\nolimits}
\newcommand{\Aut}{\mathop{\mathrm{Aut}}\nolimits}
\newcommand{\Mor}{\mathop{\mathrm{Mor}}\nolimits}
\newcommand{\Ker}{\mathop{\mathrm{Ker}}\nolimits}
\newcommand{\im}{\mathop{\mathrm{Im}}\nolimits}



% その他便利なコマンド
\newcommand{\dip}{\displaystyle} % 本文中で数式モード
\newcommand{\e}{\varepsilon} % イプシロン
\newcommand{\dl}{\delta} % デルタ
\newcommand{\pphi}{\varphi} % ファイ
\newcommand{\ti}{\tilde} % チルダ
\newcommand{\pal}{\parallel} % 平行
\newcommand{\op}{{\rm op}} % 双対を取る記号
\newcommand{\lcm}{\mathop{\mathrm{lcm}}\nolimits} % 最小公倍数の記号
\newcommand{\Probsp}{(\Omega, \F, \P)} 
\newcommand{\argmax}{\mathop{\rm arg~max}\limits}
\newcommand{\argmin}{\mathop{\rm arg~min}\limits}





% ================================
% コマンドを追加する場合のスペース 
\newcommand{\UU}{\mcal{U}}
\newcommand{\OO}{\mcal{O}}
\newcommand{\emp}{\varnothing}
\newcommand{\ceq}{\coloneqq}
\newcommand{\sbs}{\subset}
\newcommand{\mapres}[2]{\left. #1 \right|_{#2}}
\newcommand{\ded}{\hfill $\blacksquare$}
\newcommand{\id}{\mathrm{id}}
\newcommand{\isom}{\overset{\sim}{\longrightarrow}}
\newcommand{\tTop}{\textsf{Top}}


% 自前の定理環境
%   https://mathlandscape.com/latex-amsthm/
% を参考にした
\newtheoremstyle{mystyle}%   % スタイル名
    {5pt}%                   % 上部スペース
    {5pt}%                   % 下部スペース
    {}%              % 本文フォント
    {}%                  % 1行目のインデント量
    {\bfseries}%                      % 見出しフォント
    {.}%                     % 見出し後の句読点
    {12pt}%                     % 見出し後のスペース
    {\thmname{#1}\thmnumber{ #2 }\thmnote{{\normalfont (#3)}}}% % 見出しの書式

\theoremstyle{mystyle}
\newtheorem{AXM}{公理}[section]
\newtheorem{DFN}[Axiom]{定義}
\newtheorem{THM}[Axiom]{定理}
\newtheorem{PRP}[Axiom]{命題}
\newtheorem{LMM}[Axiom]{補題}
\newtheorem{CRL}[Axiom]{系}
\newtheorem{EG}[Axiom]{例}

%\newtheorem{}{Axiom}[]
\numberwithin{equation}{section} % 式番号を「(3.5)」のように印刷

\newcommand{\MM}{\mcal{M}}

% =================================





% ---------------------------
% new definition macro
% ---------------------------
% 便利なマクロ集です

% 内積のマクロ
%   例えば \inner<\pphi | \psi> のように用いる
\def\inner<#1>{\langle #1 \rangle}

% ================================
% マクロを追加する場合のスペース 

%=================================





% ----------------------------
% documenet 
% ----------------------------
% 以下, 本文の執筆スペースです. 
% Your main code must be written between 
% begin document and end document.
% ---------------------------


\begin{document}

\title{幾何学続論 (第15回)}
\author{}
\date{}

\maketitle
\begin{quote}
「今回は前回の続きと,軽くお話をして終わりましょう.」
\end{quote}
\section*{前回}
\begin{itemize}
    \item $\ast \colon \bigwedge^{k}(T_p^{\ast}M)\to \bigwedge^{m-k}(T_p^{\ast}M)$; $\omega\mapsto{\ast}\omega$ 
    \begin{align*}
        \forall\eta\in\bigwedge^{k}(T_p^{\ast}M)\quad \eta\wedge\ast\omega = g_{p}^{k}(\eta,\omega)(\omega^{1}\wedge\cdots\wedge\omega^{m})
    \end{align*}
    \item $\delta\colon\Omega^{k}(M)\to\Omega^{k-1}(M)\quad \delta\overset{\text{def}}{=}(-1)^{m(k+1)+1}\ast d\ast$.
    \item $\langle\ ,\ \rangle\colon\Omega^{k}(M)\times\Omega^{k-1}(M)\to\rr;\quad \inner<\omega,\eta>=\int_{M}\omega\wedge\eta$
\end{itemize}
\section*{今回}
\begin{Definition*}
    $\Delta\colon \Omega^{k}(M)\to\Omega^{k}(M)$:Lapracian を
    \begin{align*}
        \Delta = \delta\circ d + d\circ\delta
    \end{align*}
    で定める.

    $\omega$: harmonic form $\overset{\text{def}}{\Leftrightarrow}\Delta\omega=0$.
    
    $\mathcal{H}^{k}(M)=\left\{\omega\in\Omega^{k}(M);\ \Delta\omega=0\right\}$.
\end{Definition*}
\begin{align*}
    \inner<\Delta\omega,\omega> 
    &= \inner<\delta\circ d\omega+d\circ\delta\omega,\omega>\\
    &= \inner<\delta\circ d\omega,\omega>+\inner<d\circ\delta\omega,\omega>\\
    &= \inner<d\omega,\delta\omega>+\inner<\delta\omega,d\omega>
\end{align*}
より
\begin{align*}
    \Delta\omega=0\Longleftrightarrow \delta\omega=0\text{かつ}d\omega=0
\end{align*}
が成り立つ.よって$\mcal{H}^{k}(M)=\Ker d\cap\Ker \delta$となる.

$\omega\in\mcal{H}^{k}(M), d\eta\in\im d, \delta\mu\in\im\delta$とすれば,
\begin{align*}
    \inner<\omega,d\eta>=\inner<\delta\omega,\eta>=0,\\
    \inner<\omega,\delta\mu>=\inner<d\omega,\mu>=0,\\
    \inner<d\eta,\delta\mu>=\inner<d\circ d\eta,\mu>=0
\end{align*}
より
\begin{align*}
    \mcal{H}^{k}(M)\perp\im d, \mcal{H}^{k}(M)\perp\im \delta, \im d\perp\im\delta.
\end{align*}

「つまり,今の場合$\Omega^{k}$の中で,直和分解
\begin{align*}
    \Omega^{k}\supset\mcal{H}^{k}\oplus\im d^{k-1}\oplus\im\delta^{k+1}
\end{align*}
が成り立ってんねんな.」

実は次の定理が成り立つ.

\begin{Theorem*}[Hodge-de Rham-Kodaira]
\begin{align*}
    \Omega^{k}=\mcal{H}^{k}\oplus\im d^{k-1}\oplus\im\delta^{k+1}.
\end{align*}
この分解をHodge分解という.
\end{Theorem*}
\begin{quotation}
    
「俺は$\ldots\ldots$証明読んだことない(笑)
聞いたことあるのは,ソボレフ空間の埋め込み定理とかを使った,解析的な証明.」

\end{quotation}
この定理の系として次が従う.
\begin{Theorem*}[Hodge]
    ${}^{\forall}[\omega]\in H_{DR}^k(M)\quad {}^{\exists!}\omega_{H}\in\Omega^{k}$.
\end{Theorem*}

\begin{proof}[pf]
    $\omega=Z^{k}(M)$に対し,Hodge分解
    \begin{align*}
        \omega=\omega + d\eta + \delta\mu \quad (\omega\in\mcal{H}^{k}(M), \eta\in\Omega^{k-1}(M), \mu\in\Omega^{k+1}(M))
    \end{align*}
    を考える.
    $\omega\in Z^{k}(M)$なので
    \begin{align*}
        d\omega = d\circ\delta\mu=0. 
    \end{align*}
    従って,
    $\inner<d\circ\delta\mu,\mu>=\inner<\delta\mu,\delta\mu>=0$
    より$\delta\mu=0$. \\
    \begin{align*}
        \therefore \omega = \omega_{H}+d\eta\quad\text{となり}\quad\omega_{H}\in[\omega].
    \end{align*}
\end{proof}

\begin{quote}
    「こんなもんでどうでしょう.あとは小話.」
\end{quote}

\section*{小話}

「微分形式といえば,(俺はトポロジーの人間なので),特性類の理論がある.
また,コホモロジーといったら,群のコホモロジーがあって,
佐藤隆夫『群のコホモロジー』が最近出たが,他の和書はあまり無いと思う.
俺が専門にしてるのは写像類群で,
森田(茂之)先生のグループとか北大の秋田(利之)先生とかがやってる.
あと河澄(響矢)先生とか.」

「群から空間を作ることを考える.」

$G$: grp. \\

$X$: $K(G,1)$ sp. $\overset{\text{def}}{\Longleftrightarrow} \begin{cases}
    (\mathrm{i})\ \pi_{1}(X)\cong G\\
    (\mathrm{ii})\ \pi_{n}(X)= 0 \quad (n\geqq 2)
\end{cases}$

\begin{Problem*}
    $X$はいつ存在するか?
\end{Problem*}

\begin{Fact*}
    ${}^{\forall} G\quad {}^{\exists!} X $ up to homotopy.
\end{Fact*}
$X$を$G$のEilenberg-MacLane sp. と呼ぶ.
このとき
\begin{align*}
    H_{\ast}(G)\coloneqq H_{\ast}(K(G,1))\\
    H^{\ast}(G)\coloneqq H^{\ast}(K(G,1))
\end{align*}
とおき,群$G$の(コ)ホモロジーという.
\begin{quote}
    「$G$のEuler classも$K(G,1)$のそれで定める.」    
\end{quote}

\begin{Example*}
    $K(\zz^{n},1)=\underbrace{S^{1}\times\cdots\times S^{1}}_{n}=T^{n}$,

    $K(F_{n},1)=\bigvee_{k=1}^{n}S^1$.($F_n$はrank $n$の自由群)

    \begin{quote}
        「所謂$n$弁ブーケ.Euler数は$1-n$.」
    \end{quote}
\end{Example*}

写像類群を定義する.

$\Sigma_{g}$: oriented closed surf. ($g>1$)
に対し,
\begin{align*}
    \Mod(\Sigma_{g})\coloneqq\left\{f\colon\Sigma_{g}\to\Sigma_{g};\text{ori. pres. diffeo.}\right\}/\text{isotopy}
\end{align*}
を曲面$\Sigma_{g}$の写像類群という.

「トポロジカルなsurfaceがあると,
そこにどのくらい複素構造が入るか?というのが
リーマンのモジュライ問題.複素構造の同値類は
タイヒミュラー空間と呼ばれ,$6g-6$次元のballと同相になる.
つまり$3g-3$の複素多様体になる.
$\Mod(\Sigma_{g})$はタイヒミュラー空間に作用している
ので$\Mod(\Sigma_{g})$をモジュライ空間の基本群とも呼ぶ.
基本群を調べればモジュライ空間がよくわかることになるが,
$\Mod(\Sigma_{g})$はバカデカい群.そのためよくわかってないことも多い.
例えば忠実表現があるか?といったこと.モンスター群とも関係がある.」

$\mcal{M}_{g} = \{\text{複素構造}\}$と書く.
\begin{Fact*}
    $H^{\ast}(\mcal{M}_{g};\qq)\cong H^{\ast}(\Mod(\Sigma_{g});\qq)$
\end{Fact*}

\begin{quote}
    「この辺は特性類とかやると出てくる.」
\end{quote}

「こんなところで.今期の話で扱わなかったのは
\begin{itemize}
    \item ポアンカレ双対
    \item リーマン幾何
    \item 接続,曲率から定まるコホモロジー
    \item リー微分,フロベニウスの定理
\end{itemize}
とか.」














\end{document}