
%Don't forget to delete
%showkeys
%overfullrule
%\date \ber \er \cmt



%\documentclass[12pt,leqno]{amsart}
\documentclass[11pt, a4paper, dvipdfmx, draft]{jsarticle}


% ------------------------
% usepackage
% ------------------------
\usepackage{algorithm}
\usepackage{algorithmic}
\usepackage{amscd}
\usepackage{amsfonts}
\usepackage{amsmath}
\usepackage[psamsfonts]{amssymb}
\usepackage{amsthm}
\usepackage{ascmac}
\usepackage{color}
\usepackage{enumerate}
\usepackage{fancybox}
\usepackage[stable]{footmisc}
\usepackage{graphicx}
\usepackage{listings}
\usepackage{mathrsfs}
\usepackage{mathtools}
\usepackage{otf}
\usepackage{pifont}
\usepackage{proof}
\usepackage{subfigure}
\usepackage{tikz}
\usepackage{verbatim}
\usepackage[all]{xy}

\usetikzlibrary{cd}



% ================================
% パッケージを追加する場合のスペース 
\usepackage{latexsym}
\usepackage{wrapfig}
\usepackage{layout}
\usepackage{url}
%=================================


% --------------------------
% theoremstyle
% --------------------------
\theoremstyle{definition}


% --------------------------
% newtheoem
% --------------------------

% 日本語で定理, 命題, 証明などを番号付きで用いるためのコマンドです. 
% If you want to use theorem environment in Japanece, 
% you can use these code. 
% Attention!
% All theorem enivironment numbers depend on 
% only section numbers.
\newtheorem{Axiom}{公理}[section]
\newtheorem{Definition}[Axiom]{定義}
\newtheorem{Theorem}[Axiom]{定理}
\newtheorem{Proposition}[Axiom]{命題}
\newtheorem{Lemma}[Axiom]{補題}
\newtheorem{Corollary}[Axiom]{系}
\newtheorem{Example}[Axiom]{例}
\newtheorem{Claim}[Axiom]{主張}
\newtheorem{Property}[Axiom]{性質}
\newtheorem{Attention}[Axiom]{注意}
\newtheorem{Question}[Axiom]{問}
\newtheorem{Problem}[Axiom]{問題}
\newtheorem{Consideration}[Axiom]{考察}
\newtheorem{Alert}[Axiom]{警告}
\newtheorem{Fact}[Axiom]{事実}


% 日本語で定理, 命題, 証明などを番号なしで用いるためのコマンドです. 
% If you want to use theorem environment with no number in Japanese, You can use these code.
\newtheorem*{Axiom*}{公理}
\newtheorem*{Definition*}{定義}
\newtheorem*{Theorem*}{定理}
\newtheorem*{Proposition*}{命題}
\newtheorem*{Lemma*}{補題}
\newtheorem*{Example*}{例}
\newtheorem*{Corollary*}{系}
\newtheorem*{Claim*}{主張}
\newtheorem*{Property*}{性質}
\newtheorem*{Attention*}{注意}
\newtheorem*{Question*}{問}
\newtheorem*{Problem*}{問題}
\newtheorem*{Consideration*}{考察}
\newtheorem*{Alert*}{警告}
\newtheorem{Fact*}{事実}


% 英語で定理, 命題, 証明などを番号付きで用いるためのコマンドです. 
% If you want to use theorem environment in English, You can use these code.
%all theorem enivironment number depend on only section number.
\newtheorem{Axiom+}{Axiom}[section]
\newtheorem{Definition+}[Axiom+]{Definition}
\newtheorem{Theorem+}[Axiom+]{Theorem}
\newtheorem{Proposition+}[Axiom+]{Proposition}
\newtheorem{Lemma+}[Axiom+]{Lemma}
\newtheorem{Example+}[Axiom+]{Example}
\newtheorem{Corollary+}[Axiom+]{Corollary}
\newtheorem{Claim+}[Axiom+]{Claim}
\newtheorem{Property+}[Axiom+]{Property}
\newtheorem{Attention+}[Axiom+]{Attention}
\newtheorem{Question+}[Axiom+]{Question}
\newtheorem{Problem+}[Axiom+]{Problem}
\newtheorem{Consideration+}[Axiom+]{Consideration}
\newtheorem{Alert+}{Alert}
\newtheorem{Fact+}[Axiom+]{Fact}
\newtheorem{Remark+}[Axiom+]{Remark}

% ----------------------------
% commmand
% ----------------------------
% 執筆に便利なコマンド集です. 
% コマンドを追加する場合は下のスペースへ. 

% 集合の記号 (黒板文字)
\newcommand{\NN}{\mathbb{N}}
\newcommand{\ZZ}{\mathbb{Z}}
\newcommand{\QQ}{\mathbb{Q}}
\newcommand{\RR}{\mathbb{R}}
\newcommand{\CC}{\mathbb{C}}
\newcommand{\PP}{\mathbb{P}}
\newcommand{\KK}{\mathbb{K}}


% 集合の記号 (太文字)
\newcommand{\nn}{\mathbf{N}}
\newcommand{\zz}{\mathbf{Z}}
\newcommand{\qq}{\mathbf{Q}}
\newcommand{\rr}{\mathbf{R}}
\newcommand{\cc}{\mathbf{C}}
\newcommand{\pp}{\mathbf{P}}
\newcommand{\kk}{\mathbf{K}}

% 特殊な写像の記号
\newcommand{\ev}{\mathop{\mathrm{ev}}\nolimits} % 値写像
\newcommand{\pr}{\mathop{\mathrm{pr}}\nolimits} % 射影

% スクリプト体にするコマンド
%   例えば {\mcal C} のように用いる
\newcommand{\mcal}{\mathcal}

% 花文字にするコマンド 
%   例えば {\h C} のように用いる
\newcommand{\h}{\mathscr}

% ヒルベルト空間などの記号
\newcommand{\F}{\mcal{F}}
\newcommand{\X}{\mcal{X}}
\newcommand{\Y}{\mcal{Y}}
\newcommand{\Hil}{\mcal{H}}
\newcommand{\RKHS}{\Hil_{k}}
\newcommand{\Loss}{\mcal{L}_{D}}
\newcommand{\MLsp}{(\X, \Y, D, \Hil, \Loss)}

% 偏微分作用素の記号
\newcommand{\p}{\partial}

% 角カッコの記号 (内積は下にマクロがあります)
\newcommand{\lan}{\langle}
\newcommand{\ran}{\rangle}



% 圏の記号など
\newcommand{\Set}{{\bf Set}}
\newcommand{\Vect}{{\bf Vect}}
\newcommand{\FDVect}{{\bf FDVect}}
\newcommand{\Ring}{{\bf Ring}}
\newcommand{\Ab}{{\bf Ab}}
\newcommand{\Mod}{\mathop{\mathrm{Mod}}\nolimits}
\newcommand{\CGA}{{\bf CGA}}
\newcommand{\GVect}{{\bf GVect}}
\newcommand{\Lie}{{\bf Lie}}
\newcommand{\dLie}{{\bf Liec}}



% 射の集合など
\newcommand{\Map}{\mathop{\mathrm{Map}}\nolimits}
\newcommand{\Hom}{\mathop{\mathrm{Hom}}\nolimits}
\newcommand{\End}{\mathop{\mathrm{End}}\nolimits}
\newcommand{\Aut}{\mathop{\mathrm{Aut}}\nolimits}
\newcommand{\Mor}{\mathop{\mathrm{Mor}}\nolimits}

% その他便利なコマンド
\newcommand{\dip}{\displaystyle} % 本文中で数式モード
\newcommand{\e}{\varepsilon} % イプシロン
\newcommand{\dl}{\delta} % デルタ
\newcommand{\pphi}{\varphi} % ファイ
\newcommand{\ti}{\tilde} % チルダ
\newcommand{\pal}{\parallel} % 平行
\newcommand{\op}{{\rm op}} % 双対を取る記号
\newcommand{\lcm}{\mathop{\mathrm{lcm}}\nolimits} % 最小公倍数の記号
\newcommand{\Probsp}{(\Omega, \F, \P)} 
\newcommand{\argmax}{\mathop{\rm arg~max}\limits}
\newcommand{\argmin}{\mathop{\rm arg~min}\limits}





% ================================
% コマンドを追加する場合のスペース 
\newcommand{\UU}{\mcal{U}}
\newcommand{\OO}{\mcal{O}}
\newcommand{\emp}{\varnothing}
\newcommand{\ceq}{\coloneqq}
\newcommand{\sbs}{\subset}
\newcommand{\mapres}[2]{\left. #1 \right|_{#2}}
\newcommand{\ded}{\hfill $\blacksquare$}
\newcommand{\id}{\mathrm{id}}
\newcommand{\isom}{\overset{\sim}{\longrightarrow}}

% 自前の定理環境
%   https://mathlandscape.com/latex-amsthm/
% を参考にした
\newtheoremstyle{mystyle}%   % スタイル名
    {5pt}%                   % 上部スペース
    {5pt}%                   % 下部スペース
    {}%              % 本文フォント
    {}%                  % 1行目のインデント量
    {\bfseries}%                      % 見出しフォント
    {.}%                     % 見出し後の句読点
    {12pt}%                     % 見出し後のスペース
    {\thmname{#1}\thmnumber{ #2 }\thmnote{{\normalfont (#3)}}}% % 見出しの書式

\theoremstyle{mystyle}
\newtheorem{AXM}{公理}[section]
\newtheorem{DFN}[Axiom]{定義}
\newtheorem{THM}[Axiom]{定理}
\newtheorem{PRP}[Axiom]{命題}
\newtheorem{LMM}[Axiom]{補題}
\newtheorem{CRL}[Axiom]{系}
\newtheorem{EG}[Axiom]{例}

%\newtheorem{}{Axiom}[]
\numberwithin{equation}{section} % 式番号を「(3.5)」のように印刷

% =================================





% ---------------------------
% new definition macro
% ---------------------------
% 便利なマクロ集です

% 内積のマクロ
%   例えば \inner<\pphi | \psi> のように用いる
\def\inner<#1>{\langle #1 \rangle}

% ================================
% マクロを追加する場合のスペース 

%=================================





% ----------------------------
% documenet 
% ----------------------------
% 以下, 本文の執筆スペースです. 
% Your main code must be written between 
% begin document and end document.
% ---------------------------


\begin{document}

\title{代数曲線論}
\author{Toshi2019}

\date{Feb 20, 2022}

\maketitle
\begin{abstract}
    本稿の目的は,21年度の秋セメスターに\cite{ogs}を用いて
    行ったセミナーの内容を報告することである.
    まず複素関数論について復習してからリーマン球面について述べる.
    その後,本稿ではほとんど1次元の場合,すなわちリーマン面の場合
    しか扱わないが,一般の次元に対して複素多様体を
    定義する.セミナーで学んだ事実のうち興味あるものとして,
    リーマン・ロッホの定理をリーマン球面に対して適用したものと
    代数学の基本定理がある.これらの証明が本稿の目標である.
    最後に,特異点のない代数曲線について,
    方程式の解としての構造と複素多様体としての構造の間の
    対応について述べる.
\end{abstract}

\section*{凡例}
本稿では,次の記号について断りなく用いる.

\begin{itemize}
    \item $\zz,\qq,\rr,\cc$は整数,有理数,実数,複素数全体の集合を表す.
    \item 何らかの族$(x_i)_{i\in I}$について,添字集合が
    明らかな場合は$(x_i)_{i}$や$(x_i)$のように略記することがある.
    \item $X$を集合とする.たんに$X$の関数というときには,
    $X$上の複素数に値を取る写像とする.
    \item 差集合:集合$X$の元のうち$A$に属さないもの全体を$X-A$で表す.
    \item 球面:$S^{n-1}\coloneqq \{x\in \rr^{n}; \left\|x\right\|=1\}$
    \item 射影:直積集合に対し第$i$成分を対応させる写像を$\pr_i\colon\prod_{i} X_i\to X_i$とかく.
    \item 開近傍:距離$d$の定まっている空間$X$の点$a$に対して$U_r(a)=\{x\in X; d(x,a)<r\}$とかく.
    \item ワイもや:$X$と$Y$を位相空間とし$f\colon X\to Y$を
    連続写像とする.$X$の部分集合$A$が連結(コンパクト)ならば$f$による
    像$f(A)$も連結(コンパクト)である.この性質を `ワイもや' と呼ぶ.
\end{itemize}
\section{複素関数論}

複素関数論について復習する
\footnote{複素関数論については例えば\cite{jimbo,yoshida,kaneko,fuji}を参照.}.
$c$を複素数とし$f$をガウス平面の開集合$\varOmega$で
定義された複素数に値を取る関数とする.
$f$が$z=c$で解析的であるとは,
何らかの実数$r>0$に対しては,$U_r(c)$で
巾級数$\dip \sum_{n=0}^{\infty}a_{n}(z-c)^{n}$を用いて
$\dip f(z)=\sum_{n=0}^{\infty}a_{n}(z-c)^{n}$と表せる
ことをいう.
\begin{Definition}
    $f$をガウス平面の開集合$\varOmega$で
    定義された関数とし,$z=x+iy$を$\varOmega$に属する複素数とする.
    コーシー・リーマンの方程式
    \begin{align}
        \frac{1}{2}\left(\frac{\p}{\p x}+i\frac{\p}{\p y}\right)f(z)=0
    \end{align}
    をみたす$C^1$級の関数を正則関数という.
\end{Definition}
$f$が$z=c$で解析的となるのは,$z=c$で(複素)微分可能となるときである.
またこのとき$f$はコーシー・リーマンの方程式をみたす
\footnote{正則関数の特徴づけについては,例えば\cite[4.3節]{kaneko}を参照.}.


\section{リーマン面}

\subsection{リーマン球面}

$\cc^{2}$から原点$0=(0,0)$を除いた集合$\cc^{2}-\{0\}$
の点$(a_{0},a_{1}), (b_{0},b_{1})$に対し次の関係を考える.
\begin{align}\label{eq:sim1}
    (a_{0},a_{1})\sim (b_{0},b_{1})
    \Longleftrightarrow
    (a_{0},a_{1})= c\cdot(b_{0},b_{1})
    \text{となる複素数}c\neq0\text{が存在する.}
\end{align}
これは同値関係である.
$(a_0,a_1)$の同値類$\{c\cdot(a_0,a_1); c\in\cc-\{0\}\}$を
$[a_0\colon a_1]$とかく.
\begin{proof}[\eqref{eq:sim1}\textbf{が同値関係になることのチェック}]
    
    (反射律)
    $c=1$は$(a_{0},a_{1})= 1\cdot(a_{0},a_{1})$をみたす.

    (対称律)
    複素数$c\neq0$を$(a_{0},a_{1})= c \cdot (b_{0},b_{1})$
    をみたすものとすると,複素数$c^{-1}\neq0$は
    $(b_{0},b_{1})=c^{-1} \cdot(a_{0},a_{1})$
    をみたす.

    (推移律)
    複素数$c,c'\neq0$をそれぞれ
    $(a_{0},a_{1})= c \cdot (b_{0},b_{1})$,
    $(b_{0},b_{1})= c' \cdot (c_{0},c_{1})$
    をみたすものとする.このとき複素数$cc'\neq0$は
    \begin{align*}
        (a_{0},a_{1})
        = c \cdot (b_{0},b_{1})
        =cc' \cdot (c_{0},c_{1})
    \end{align*}
    をみたす.
\end{proof}

同値関係${\sim}$の定める商写像を用いて次の集合を定義する.

\begin{Definition}
    $\pp^{1}(\cc) = \left(\cc^{2}-\{0\}\right)/{\sim}$
    を\textbf{リーマン球面}(Riemann sphere) という.
    $\pp^{1}(\cc)$を$\pp^{1}_{\cc}$とか$\pp^{1}$ともかく\footnote{トポロジストは$\cc\pp^1$等と書くらしい.}.
\end{Definition}

$\pp^1$の任意の点$P$は$[a_0\colon a_1]$の形に表せる.
実際,$P$を$\pp^1$の点とすると,$\pp^1$の
定義より,$(a_{0},a_{1})\in \cc^{2}-\{0\}$で
$P=\pi(a_{0},a_{1})=[a_0\colon a_1]$となるものが存在する.

また,$[a_0\colon a_1]=[b_0\colon b_1]$となるのは,
$a_0\colon a_1=b_0\colon b_1$となるときである.
実際,
\begin{align*}
    [a_0\colon a_1]=[b_0\colon b_1]
    &\Longleftrightarrow
    (a_0, a_1)\in[b_0\colon b_1]\\
    &\Longleftrightarrow
    (a_0, a_1)\sim(b_0, b_1)\\
    &\Longleftrightarrow
    (a_0, a_1)=c(b_0, b_1) \text{ for some }c\neq0\\
    &\Longleftrightarrow
    a_0=cb_0, a_1=cb_1 \text{ for some }c\neq0\\    
    &\Longleftrightarrow
    a_0\colon b_0 = a_1 \colon b_1\\
    &\Longleftrightarrow
    a_0 b_1 = a_1 b_0\\
    &\Longleftrightarrow
    a_0\colon b_1 = b_0 \colon b_1
\end{align*}
である.

\begin{Definition}\label{def:coord1}
    次の写像の組を考える.
    $\begin{tikzcd}
      {\cc^{2}-\{0\}}
        \arrow[r, shift left ,"\pr_1=X_0"]
        \arrow[r, shift right,"\pr_2=X_1"']
      & {\cc}
    \end{tikzcd}; (a_0,a_1)\mapsto a_0,a_1.$
    この組を$\cc^{2}-\{0\}$の標準座標,$\pp^1$の同次座標という.
\end{Definition}
$P\in\pp^1$を代表する$\cc^{2}-\{0\}$の点$\widetilde{P}$の
標準座標の値$(a_0,a_1)$が$P$の同次座標の値である.
なお,$P$に対する$\widetilde{P}$の取り方,
すなわち$(a_0,a_1)$の取り方には
任意性がある.

$\pp^{1}$は商写像$\pi \colon \cc^{2}-\{0\}
\longrightarrow\left(\cc^{2}-\{0\}\right)/{\sim}$
による商位相により位相空間になる.この定義から$\pi$の連続性が従う.

$\pp^1$の位相空間としての性質を調べるために,次の部分集合を定義する.
\begin{align*}
    U_0=\{[a_0\colon a_1]\in\pp^1; a_0\neq0\},\\
    U_1=\{[a_0\colon a_1]\in\pp^1; a_1\neq0\}.
\end{align*}
このとき次が成り立つ.
\begin{align}
    U_0\cup U_1 &= \pp^1, \label{eq:cov1}\\
    U_0\cap U_1 
    &= \{[a_0\colon a_1]\in\pp^1; a_0, a_1\neq0\} \label{eq:inter-p1}\\
    &= U_0 - \{[1\colon 0]\}  \notag\\
    &= U_1 - \{[0\colon 1]\}. \notag
\end{align}

\begin{Lemma}\label{mnf:p1}
    1. 
    商写像$\pi \colon \cc^{2}-\{0\}
    \longrightarrow\left(\cc^{2}-\{0\}\right)/{\sim}$
    は開写像である.

    2. 
    $U_0$と$U_1$は$\pp^1$の開集合であり,
    \begin{align*}
        \pphi_0&\colon U_0\overset{{\sim}}{\longrightarrow}\cc;\ [a_0\colon a_1]\mapsto a_1/a_0,\\
        \pphi_1&\colon U_1\overset{{\sim}}{\longrightarrow}\cc;\ [a_0\colon a_1]\mapsto a_0/a_1
    \end{align*}
    はともに同相写像である.

    3. 
    任意の$\dip A = \begin{bmatrix}
        a&b\\c&d
    \end{bmatrix}\in GL(2,\cc)$は自己同相写像
    \begin{align*}
        p_A\colon \pp^1\overset{{\sim}}{\longrightarrow}\pp^1;
        \begin{bmatrix}
            a_0\\a_1
        \end{bmatrix}
        \mapsto
        \begin{bmatrix}
            a&b\\c&d
        \end{bmatrix}
        \begin{bmatrix}
            a_0\\a_1
        \end{bmatrix}
    \end{align*}
    を引き起こす.

    4. 
    $\pp^1$は第2可算公理をみたす連結なコンパクトハウスドルフ空間である.
\end{Lemma}

\begin{proof}[\textbf{証明}]
    1. 
    $U$を$\cc^2-\{0\}$の開集合とする.$\pi(U)$が$\pp^1$の
    開集合であること,すなわち$\pi^{-1}(\pi(U))$が$\cc^2-\{0\}$の
    開集合であることを示す.
    いま,任意の開集合$U\subset\cc^2-\{0\}$に対し,
    複素数$c\neq0$を用いて
    \begin{align*}
        cU = \left\{(ca_0,ca_1); (a_0,a_1)\in\cc^2-\{0\}\right\}
    \end{align*}
    とおくと,$cU$は$\cc^2-\{0\}$の開集合であり,
    \begin{align}\label{eq:proj}
        \pi^{-1}(\pi(U)) = \bigcup_{c\in\cc-\{0\}} cU \tag{$\ast$}
    \end{align}
    なので,$\pi^{-1}(\pi(U))$は$\cc^2-\{0\}$の
    開集合である.

    2. 
    まず$U_0, U_1$が$\pp^1$の開集合であることを示す.
    $U_0=\{[a_0:a_1]; a_0\neq0\}$は$V_0=\{(a_0,a_1);a_0\neq0\}$
    の$\pi$による像であり,$V_0$は
    $\cc^2-\{0\}$の開集合であるから,
    $U_0$は$\pp^1$の開集合である.同様に$U_1$も$\pp^1$の開集合である.

    $\pphi_0\colon U_0\to\cc$が連続であることを示す.$V$を$\cc$の
    開集合とする.
    $V=\pphi_0\circ\pi(V_0) (= \widetilde{\pphi_0}(V_0)$
    とおく)である.
    $\widetilde{\pphi_0}^{-1}(V) 
    = \pi^{-1}\left(\pphi_0^{-1}(V)\right)$は$V_0$の開集合である.
    したがって,これは$\cc^2-\{0\}$の開集合であり,商位相の定義から$\pphi_0^{-1}(V)\subset U_0$
    は開集合である.
    \begin{equation*}
        \vcenter{\xymatrix@C=36pt@R=36pt{
        V_0 \ar@{{}->>}[d]_{\pi} \ar@{{}->}[rd]^-{\widetilde{\pphi_0}} \\
        U_0 \ar[r]_{\pphi_0}  
        & V\ar@{}[lu]
        }}
    \end{equation*}
    
    $\pphi_0$が同相であることを示す.$\psi_0\colon \cc \to U_0$を
    $\psi_0(z)=[1\colon z]$で定める.このとき
    $\psi_0\circ\pphi_0\left([a_0\colon a_1]\right)
    =\psi_0\left( a_1/a_0 \right)
    =[1\colon a_1/a_0]=[a_0\colon a_1]$である.
    また$\pphi_0\circ\psi_0(z)=\pphi_0([1\colon z])=z/1=z$.
    したがって,$\psi_0\circ\pphi_0=\id_{U_0}$かつ
    $\pphi_0\circ\psi_0=\id_{\cc}$であり,$\psi_0=\pphi_0^{-1}$である.
    $\psi_0=\pphi_0^{-1}$は自然な単射$\cc 
    \hookrightarrow \cc^2-\{0\}$と$\pi$の合成であり,
    これらは連続なので,その合成である$\psi_0$も連続である.
    以上より$\pphi_0$は同相である.

    3. 
    $\dip A = \begin{bmatrix}
        a&b\\c&d
    \end{bmatrix}$を可逆な行列とする.
    $A$を自己同形$\cc^2\to\cc^2$とみたとき,
    それを$\cc^2-\{0\}$に制限した
    $\mapres{A}{\cc^2-\{0\}}\colon\cc^2-\{0\}\to\cc^2-\{0\}$
    は自己同相であり,
    逆写像は$\mapres{A^{-1}}{\cc^2-\{0\}}$で与えられる.
    一般に$A(cx)=cAx$なので,$A$から可逆な写像$p_A$が不備なく定まり,
    逆写像は$p_{A^{-1}}$で与えられる.

    $p_A$が連続であることを示す.$V$を$\pp^1$の開集合とする.
    次の図式が可換であり,$\pi$と$A$は連続写像であるから,
    $\pi^{-1}\left(p_A^{-1}(V)\right)
    =A^{-1}\left(\pi^{-1}(V)\right)$は$\cc^2-\{0\}$の開集合
    である.
    \begin{equation*}
        \vcenter{\xymatrix@C=36pt@R=36pt{
        \cc^2-\{0\} 
        \ar@{{}->}[d]_{\pi} 
        \ar@{{}->}[r]^-{A} 
        & \cc^2-\{0\} 
        \ar@{{}->}[d]^{\pi} 
        \\
        \pp^1 \ar[r]_{p_A}  
        & \pp^1 \ar@{}[lu]
        }}
    \end{equation*}
    $\pp^{1}$の商位相の定義より$\pi^{-1}(V)$は$\pp^{1}$の開集合である.
    したがって$p_A$で連続である.$p_A^{-1}$が連続であることも同様である.

    4. 
    第2可算公理をみたすこと:
    \begin{align*}
        \qq(\sqrt{-1})=\{a+b\sqrt{-1};a,b\in\qq\}
    \end{align*}
    に属する点$z$と有理数$p$に対し
    $U_{p}(z)$を考えると
    $\left(U_{p}(z)\right)_{p\in\qq,z\in\cc}$
    は$\cc$の位相空間としての基底になる.
    したがって$\cc$は第2可算公理をみたす.
    直積集合$\cc^2$も第2可算であるから,1点を除いた$\cc^2-\{0\}$もそうであり,
    これに全射$\pi$を適用した$\pp^1$も第2可算公理をみたす.

    連結かつコンパクトであること:
    $S^3=\{P=(a_0,a_1)\in\cc^2;|a_0|^2+|a_1|^2=1\}\subset\cc^2-\{0\}$
    であり,$\cc-\{0\}$の相対位相により,$S^3$は有界閉集合
    つまりコンパクト集合であり,連結である.
    全射連続写像$\mapres{\pi}{S^3}\colon S^3\to\pp^1$により 
    `ワイもや' で$\pp^1$は連結かつコンパクト.
    $\mapres{\pi}{S^3}$が全射であることは
    \begin{align*}
        [a_0\colon a_1]
        =
        \left[
            \frac{a_0}{\sqrt{a_0^2+a_1^2}}\colon \frac{a_1}{\sqrt{a_0^2+a_1^2}}
        \right]
    \end{align*}
    であることからしたがう.

    ハウスドルフであること:
    $P\neq Q$を$\pp^1$の点とする.
    $p\colon GL(2,\cc)\to \Aut(\pp^1)$は
    全射.したがって,$U_0\subset \pp^1$から,任意の
    $p_{A}\in \Aut(U_0)$に対し$A\in GL(2,\cc)$が存在する.
    つまり$p_A(P),p_A(Q)\in U_0$となる$A\in GL(2,\cc)$が存在する.
    $U_0\cong \cc$であり$\cc$はハウスドルフなので,$p_A(P)$の
    開近傍$U_P$と$p_A(Q)$の開近傍$U_Q$で$U_P\cap U_Q=\emp$を
    みたすものが存在する.
    $U_P$と$U_Q$は$U_0\subset\pp^1$の開集合であり,$p_{A}$が
    同相なので$p_A^{-1}(U_P)$, $p_A^{-1}(U_Q)$は$\pp^1$に
    おける$P$, $Q$の開近傍で
    $p_A^{-1}(U_P)\cap p_A^{-1}(U_Q)=\emp$
    をみたす.よって$\pp^1$はハウスドルフである.
\end{proof}

\begin{Attention}[{\eqref{eq:proj}}の幾何的イメージ]
    この等式については次の図\ref{fig:prj1}を見ると理解しやすい.
    \begin{figure}[ht]
        \centering
        \scalebox{0.8}{
        \begin{tikzpicture}
            \draw[>=stealth,semithick] (-2,0)--(3,0); %x軸
            \draw[>=stealth,semithick] (0,-2)--(0,4); %y軸
            \draw (0,0)node[below right]{O}; %原点
            \draw[thick, domain=-2:3] plot(\x,{0.5*\x});
            \draw[thick, domain=-1.:2] plot(\x,2*\x);
            \fill[black!20] (1.5,3) rectangle (2.5,1.25); %四角2U
            \draw[line width = 0.5pt] (1.5,3) rectangle (2.5,1.25); %四角2U
            \draw (2.5,1.25)node[below]{$2U$}; %点(2.5,1.25)
            \fill[black!20] (1,2) rectangle (1.66,0.83); %四角Uずらし
            \draw[line width = 0.5pt] (1,2) rectangle (1.66,0.83); %四角Uずらし
            \fill[black!40] (0.75,1.5) rectangle (1.25,0.625); %四角U
            \draw[line width = 0.5pt] (0.75,1.5) rectangle (1.25,0.625); %四角U
            \draw (0.75,1.5)node[left]{$U$}; %点(2.5,1.25)\draw[line width=1pt] (0.75,1.5) rectangle (1.25,0.625); %四角
            \fill[black!20] (-0.75,-1.5) rectangle (-1.25,-0.625); %四角-U
            \draw[line width = 0.5pt] (-0.75,-1.5) rectangle (-1.25,-0.625); %四角-U
            \fill[black!20] (-1,-2) rectangle (-1.66,-0.83); %四角-U
            \draw[line width = 0.5pt] (-1,-2) rectangle (-1.66,-0.83); %四角-U
            \draw (-0.75,-1.5)node[right]{$-U$}; %点(2.5,1.25)
            \fill[black!20] (0.375,0.75) rectangle (0.625,0.3125); %四角(1/2)U
            \draw[line width = 0.5pt] (0.375,0.75) rectangle (0.625,0.3125); %四角(1/2)U
            \draw (0.625,0.2)node[right]{$(1/2)U$}; %点(2.5,1.25)
            \draw[very thick, domain=-2.:3] plot(\x,{0.8*\x});
            %\draw[very thick, domain=-1.7:2.7] plot(\x,{1.3*\x});
            \draw (1.8,3.6)node[right]{$U$を通る直線の上端};
            \draw (3,2.4)node[right]{$U$を通る直線};
            \draw (3,1.5)node[right]{$U$を通る直線の下端};
        \end{tikzpicture}}
        \caption{商写像の逆像1}
        \label{fig:prj1}
    \end{figure}
    いま,簡単のため$\rr^2$の部分集合$U$について考える.
    $\pi(U)$は$U$を通る直線たちの集合である.像が$\pi(U)$となる
    ようなもの,つまり$\pi^{-1}\left(\pi(U)\right)$は$U$を$c$倍に
    拡大・縮小したもの全体である.したがって
    \begin{align*}
        \pi^{-1}\left(\pi(U)\right) 
        &= (U\text{を射影したものと同じところに行くもの})\\
        &= (U\text{を}c\text{倍に拡大したもの全て})\\
        &= \bigcup_{c\in\rr-\{0\}}cU
    \end{align*}
    のようになり,$\cc^2$の場合には\eqref{eq:proj}が成り立つ.

    次のような場合も見ておくと複素トーラスの導入のときなどに役立つ.
    \begin{align*}
        \pi\colon \rr \twoheadrightarrow \rr/\zz; x\mapsto x\mod{\zz}
    \end{align*}
    を考える.これは図\ref{fig:prj2}の上の直線を螺旋状に巻いて潰し,1次元トーラス
    にしたものである.

    \begin{figure}[hb]
        \centering
        \scalebox{0.8}{\begin{tikzpicture}[x=1cm]
            \fill[black!40] (0.7,0.1) rectangle (1.3,-0.1);
            \draw (1.3,-0.1) node[below]{$U$};
            \draw[->,>=stealth,semithick] (-5,0)--(5,0)node[below]{$\rr$}; %x軸
            \foreach \x in {-5, -4,...,4} \draw (\x,-0.2)--(\x,0.2);
            
            \draw[->,>=stealth,semithick] (0,-1)--(0,-2); %射影

            \draw[black!40,line width=5pt] (0,-5) arc (270:330: 3 and 1);
            \draw (0,-1.5) node[right]{$\pi$};
            \draw (0,-4) ellipse (3 and 1);
            \draw (1,-5.2)--(1,-4.8);
            \draw (1.3,-5.2) node[right]{$\pi(U)$};
            \draw[->,>=stealth,semithick] (0,-5.5)--(0,-6.5); %射影
            \draw (0,-6) node[right]{$\pi(U)$に来るものを集める};

            \foreach \x in {-4,-3,...,4} \fill[black!40] ({\x-0.3},-6.9) rectangle ({\x+0.3},-7.1);
            \draw (1.2,-7.1) node[below]{$U$};
            \draw (2.1,-7.1) node[below]{$U+1$};
            \draw (3.1,-7.1) node[below]{$U+2$};
            \draw (4.1,-7.1) node[below]{$U+3$};
            \draw (0.1,-7.1) node[below]{$U-1$};
            \draw (-0.9,-7.1) node[below]{$U-2$};
            \draw (-2,-7.1) node[below]{$U-3$};
            \draw (-3,-7.1) node[below]{$U-4$};
            \draw (-4,-7.1) node[below]{$U-5$};

            \draw[->,>=stealth,semithick] (-5,-7)--(5,-7)node[below]{$\rr$}; %x軸
            \foreach \x in {-5, -4,...,4} \draw (\x,-6.8)--(\x,-7.2);
        \end{tikzpicture}}
        \caption{商写像の逆像2}
        \label{fig:prj2}
    \end{figure}
    この場合は次のようになる.
    \begin{align*}
        \pi^{-1}\left(\pi(U)\right) 
        &= (U\text{を射影したものと同じところに行くもの})\\
        &= (U\text{を整数分だけずらしたもの全て})\\
        &= \bigcup_{n\in\zz}(U+n).
    \end{align*}
\end{Attention}

\subsection{貼りあわせ関数}\label{ssec:patch}

補題\ref{mnf:p1}.2 から
$\pphi_0 \colon U_0\isom\cc$, $\pphi_1\colon {U_1} \isom\cc$
である.
ここで,$\pphi_0(U_0)$の標準座標を$w$, 
$\pphi_1(U_1)$の標準座標を$z$で表すことにする.
定義\ref{def:coord1}のようにかくと
\begin{align*}
    z&\colon \pphi_1(U_1)= \cc \to \cc; (a)\mapsto a\\
    w&\colon \pphi_0(U_0)= \cc \to \cc; (b)\mapsto b
\end{align*}
のようになる.複素数の一つ組に対し第一成分を対応させるということである.
これによって点$(a)$と座標値$z(a)$を同一視し,点を単に$z$と書いたりする.
ガウス平面$\cc$に,そこでの標準座標をつけて$\cc_z$, $\cc_w$のように表すと,
$\cc_w\subset\pp^1$, $\cc_z\subset\pp^1$とみなせる.
例えば$\cc_z$の点$1+\sqrt{-1}$は$\pp^1$では
\begin{align*}
    \pphi_1^{-1}\left(1+\sqrt{-1}\right)=\left[1+\sqrt{-1}\colon 1\right]    
\end{align*}
であり
$\cc_w$の点$1/(1+\sqrt{-1})$は$\pp^1$では
\begin{align*}
    \pphi_0^{-1}\left(\frac{1}{1+\sqrt{-1}}\right)
    =\left[1\colon \frac{1}{1+\sqrt{-1}}\right]
    =\left[1+\sqrt{-1}\colon 1\right]        
\end{align*}
である.
本節では,$\cc_z$と$\cc_w$の間の関係を調べる.

いまの$1+\sqrt{-1}$の例を見ると
\begin{align}\label{eq:patch1}
    z=\frac{1}{w}
\end{align}
の関係が成り立っているようである.
他の例も見る.
例えば
点$[2\colon 1]$と点$[1\colon 1/2]$は$\pp^1$では同じものである.
これらをそれぞれ
$\pphi_1$, $\pphi_0$で$\cc_z$, $\cc_w$の点とみなすと
座標値は$z=2$と$w=1/2$である.
したがって$z=1/w$が成り立つ.

$z=0$のときはうまくいかない.
$[z\colon 1] = [1\colon 1/z] = [1\colon w]$
のようにして$w=1/z$となるものを見つけたいが$w=1/0$と
なってしまい不合理である.
$w=0$のときも同様である.
そもそも$\pphi_0$は$U_0$上で定義されており,
$z=0$となる$[0\colon 1]$のような点に対しては$w$の値は定まっていない.
つまり,関係式\eqref{eq:patch1}が成り立つためには
$z$も$w$も0でないことが必要である.
逆に$z$と$w$のどちらも0でなければ,
関係式\eqref{eq:patch1}が成り立つ.

$z,w\ne 0$は\eqref{eq:inter-p1}より
$[z\colon w]\in U_0\cap U_1$ということである.
$[z\colon w]\in U_0\cap U_1$のとき$z$は$w$の正則関数になっている.
$\pphi_0 (U_0\cap U_1) = \cc_w-\{0\} 
= \cc_w \cap \cc_z 
= \cc_z-\{0\} 
= \pphi_1 (U_0\cap U_1)$
なので,
この正則関数を
$\pphi_{10}\colon \cc_w-\{0\}=\pphi_0 (U_0\cap U_1) 
\to \cc_z-\{0\} = \pphi_1 (U_0\cap U_1)$
とかくことにすると,次の図式が可換になる.
\begin{equation*}
    \vcenter{\xymatrix@C=35pt@R=35pt{
    {U_0\cap U_1} 
        \ar@{<-{}}[d]^{\pphi_0^{-1}} 
        \ar@{{}=}[r]^-{[1\colon w]=[z\colon 1]}
    & {U_0 \cap U_1} 
        %
        \\
    \cc_w -\{0\}
        \ar[r]_{\pphi_{10}}
    & \cc_z -\{0\}
        \ar@{<-{}}[u]_{\pphi_1}
        %\ar@{}[lu]
    }}
\end{equation*}
つまり,$\pphi_{10} = \pphi_1\circ \pphi_0^{-1}$である.
この正則関数を貼りあわせ関数と呼ぶ.
また,$\pphi_{01} = \pphi_0\circ \pphi_1^{-1}\colon \pphi_1 (U_0\cap U_1) \to \pphi_0 (U_0\cap U_1)$
も$w=1/z$として同様に定まる.これは正則であり$\pphi_{10}$の逆関数でもある.

\subsection{複素多様体とリーマン面}

$\cc^{n}$での座標が$z=(z_{1},\dots,z_{n})$であるとき
複素数空間$\cc^{n}$を
$\cc^{n}_{z}$とか$\cc^{n}_{(z_{1},\dots,z_{n})}$とかく.

\begin{Definition}[$n$次元複素多様体]
    $X$を位相空間とする.
    $(\pphi_{i}\colon U_{i}\to \UU_{i})_{i\in I}$を
    写像の族とする.このとき,対$(X,(\pphi_{i})_{i})$が
    次の条件(1)--(4)をみたすとき,$X$を台集合とし$(\pphi_{i})_i$を
    座標近傍系とする\textbf{$n$次元複素多様体} ($n$-dimensional complex manifold) という.

    (1) 
    $X$は空集合でなく,第2可算公理を満たす連結なハウスドルフ空間である\footnote{
        条件 (1) のうち第2可算と連結を課さないことも多い (\cite{kobayashi1}など).}.

    (2) 
    すべての$i\in I$に対して$U_i$は$X$の空でない開集合であり,
    $(U_i)_i$は$X$の開披覆である.

    (3) 
    すべての$i\in I$に対して,
    $\UU_i$は$\cc^{n}_{(z_{1},\dots,z_{n})}$の
    空でない開集合であり$\pphi_{i}\colon U_i\to\UU_i$は同相である.

    (4) 
    任意の$i\neq j \in I$で$U_i\cap U_j \neq \emp$をみたすもの
    に対して$\UU_{ij}\ceq \pphi_j(U_i\cap U_j)\sbs \UU_j$と
    おくとき,$\pphi_{ij}\ceq 
    \mapres{\pphi_{i}\circ\pphi_{j}^{-1}}{\UU_{ij}}
    \colon 
    \UU_{ij}\to\UU_{ji}$は正則である.
\end{Definition}

1次元複素多様体を\textbf{リーマン面} (Riemann surface) という.

\begin{Proposition}
    リーマン球面$\pp^1$はリーマン面である.    
\end{Proposition}

\begin{proof}[\bf{証明}]
    (1) 
    補題\ref{mnf:p1}.4 からしたがう.

    (2) 
    \eqref{eq:cov1}と\eqref{eq:inter-p1} からしたがう.

    (3) 
    補題\ref{mnf:p1}.2 からしたがう.

    (4) 
    \ref{ssec:patch}節で説明した.
\end{proof}

\section{1次元における諸結果}

% $\pp^{1}$の定義
% 有理形関数 と 極の位数

次はリーマンロッホの定理の特殊な場合である.

\begin{Theorem}[{\cite[命題1.14]{ogs}}]
    $P_{1},\dots,P_{m}$を$\pp^{1}$の互いに相異なる点とする.
    $\pp^{1}$上の有理形関数で各$P_i$のみで高々$n_i$の極をもつもの全体
    \begin{align*}
        \Gamma\left(\pp^{1}, \OO_{\pp^{1}}\left(\sum_{i=1}^{m}n_{i}P_{i}\right)\right)
    \end{align*}
    は関数の和とスカラー倍によって$\cc$ベクトル空間を成す.
    さらにその空間の次元は次で与えられる.
    \begin{align*}
        \dim\Gamma\left(
            \pp^{1}, \OO_{\pp^{1}}\left(
                \sum_{i=1}^{m}n_{i}P_{i}
                \right)
            \right)
        = 1+\sum_{i=1}^{m}n_{i}.
    \end{align*}
\end{Theorem}


\section{コンパクトリーマン面}

\begin{Theorem}[代数学の基本定理{\cite[系2.24]{ogs}}]
    $n\geqq1$とする.複素数を係数とする$n$次方程式
    \begin{align*}
        a_{0}z^{n}+a_{1}z^{n-1}+\dots +a_{n}=0,\quad a_{0}\neq 0
    \end{align*}
    に対し,複素数解$\alpha$が存在する.
\end{Theorem}

\section{それから}
ここはかくか微妙
\section{結語}
本稿はセミナーの予習ノートと板書(後述)を元に書いた.
この辺りの経緯について述べる.
2021年度は対面での課外活動が制限されていた.
そのため,Zoom を用いた遠隔でのセミナーを行うことが主であった.
板書には Microsoft OneNote を利用した.
%Microsoft 365 Apps for Enterprise で提供されており,
%立命館の学生は無料で利用できる(\cite{IT}参照).
Whiteboard Fox と異なり,14日間で板書が削除されることもないので
便利である.
ただし,時々動作が重くなることがあったので,
発表者の画面共有もしながらセミナーを行った.
この発表の際に書いた板書と予習ノートが本稿の下書きに当たる.
清書するにあたって行ったのは,
%一般的な事柄を述べてから特殊なものが演繹されるように
叙述の順序を変更したことと,
セミナー中に気付いたことなどのコメントを加えたことである.
%セミナーは筆者と谷川遼太郎君の2名で行った.
%本稿で述べた事柄には谷川君のアイデアも取り入れられていることを注意しておく.




\begin{thebibliography}{15}

    \bibitem{ueno} 上野健爾, 代数幾何, 岩波書店, 2005.

    \bibitem{umemura} 梅村浩, 楕円関数論 楕円曲線の解析学 [増補新装版], 東京大学出版会, 2020.

    \bibitem{ogs} 小木曽啓示, 代数曲線論, 朝倉書店, 2002.

    \bibitem{kaneko} 金子晃, 関数論講義 (ライブラリ数理・情報系の数学講義, 5), サイエンス社, 2021.

    \bibitem{kobayashi1} 小林昭七, 複素幾何1 (岩波講座現代数学の基礎, 29), 岩波書店, 1997.

    \bibitem{jimbo} 神保道夫, 複素関数論 (岩波講座現代数学への入門, 4), 岩波書店, 1995.

    \bibitem{takebe} 武部尚志, 楕円積分と楕円関数 おとぎの国の歩き方, 日本評論社, 2019.

    \bibitem{fuji} 藤本坦孝, 複素解析 (岩波講座現代数学の基礎, 3), 岩波書店, 1996.

    \bibitem{yoshida} 吉田洋一, 函数論 第2版 (岩波全書, 141), 岩波書店, 1973.

    %\bibitem{IT} 個人のOfficeの利用について(Microsoft 365 Apps for Enterprise)\url{https://it.support.ritsumei.ac.jp/hc/ja/articles/900006824883}, 2022年3月5日 最終閲覧.

    %\bibitem{tmp} 著者, 書名 (シリーズ名, 巻数), 発行所, 発行都市名, 年号
\end{thebibliography}

\end{document}