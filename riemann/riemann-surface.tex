
%Don't forget to delete
%showkeys
%overfullrule
%\date \ber \er \cmt



%\documentclass[12pt,leqno]{amsart}
\documentclass[11pt, a4paper, dvipdfmx, draft]{jsarticle}


% ------------------------
% usepackage
% ------------------------
\usepackage{algorithm}
\usepackage{algorithmic}
\usepackage{amscd}
\usepackage{amsfonts}
\usepackage{amsmath}
\usepackage[psamsfonts]{amssymb}
\usepackage{amsthm}
\usepackage{ascmac}
\usepackage{color}
\usepackage{enumerate}
\usepackage{fancybox}
\usepackage[stable]{footmisc}
\usepackage{graphicx}
\usepackage{listings}
\usepackage{mathrsfs}
\usepackage{mathtools}
\usepackage{otf}
\usepackage{pifont}
\usepackage{proof}
\usepackage{subfigure}
\usepackage{tikz}
\usepackage{verbatim}
\usepackage[all]{xy}

\usetikzlibrary{cd}



% ================================
% パッケージを追加する場合のスペース 
\usepackage{latexsym}
\usepackage{wrapfig}
\usepackage{layout}
%=================================


% --------------------------
% theoremstyle
% --------------------------
\theoremstyle{definition}


% --------------------------
% newtheoem
% --------------------------

% 日本語で定理, 命題, 証明などを番号付きで用いるためのコマンドです. 
% If you want to use theorem environment in Japanece, 
% you can use these code. 
% Attention!
% All theorem enivironment numbers depend on 
% only section numbers.
\newtheorem{Axiom}{公理}[section]
\newtheorem{Definition}[Axiom]{定義}
\newtheorem{Theorem}[Axiom]{定理}
\newtheorem{Proposition}[Axiom]{命題}
\newtheorem{Lemma}[Axiom]{補題}
\newtheorem{Corollary}[Axiom]{系}
\newtheorem{Example}[Axiom]{例}
\newtheorem{Claim}[Axiom]{主張}
\newtheorem{Property}[Axiom]{性質}
\newtheorem{Attention}[Axiom]{注意}
\newtheorem{Question}[Axiom]{問}
\newtheorem{Problem}[Axiom]{問題}
\newtheorem{Consideration}[Axiom]{考察}
\newtheorem{Alert}[Axiom]{警告}
\newtheorem{Fact}[Axiom]{事実}


% 日本語で定理, 命題, 証明などを番号なしで用いるためのコマンドです. 
% If you want to use theorem environment with no number in Japanese, You can use these code.
\newtheorem*{Axiom*}{公理}
\newtheorem*{Definition*}{定義}
\newtheorem*{Theorem*}{定理}
\newtheorem*{Proposition*}{命題}
\newtheorem*{Lemma*}{補題}
\newtheorem*{Example*}{例}
\newtheorem*{Corollary*}{系}
\newtheorem*{Claim*}{主張}
\newtheorem*{Property*}{性質}
\newtheorem*{Attention*}{注意}
\newtheorem*{Question*}{問}
\newtheorem*{Problem*}{問題}
\newtheorem*{Consideration*}{考察}
\newtheorem*{Alert*}{警告}
\newtheorem{Fact*}{事実}


% 英語で定理, 命題, 証明などを番号付きで用いるためのコマンドです. 
% If you want to use theorem environment in English, You can use these code.
%all theorem enivironment number depend on only section number.
\newtheorem{Axiom+}{Axiom}[section]
\newtheorem{Definition+}[Axiom+]{Definition}
\newtheorem{Theorem+}[Axiom+]{Theorem}
\newtheorem{Proposition+}[Axiom+]{Proposition}
\newtheorem{Lemma+}[Axiom+]{Lemma}
\newtheorem{Example+}[Axiom+]{Example}
\newtheorem{Corollary+}[Axiom+]{Corollary}
\newtheorem{Claim+}[Axiom+]{Claim}
\newtheorem{Property+}[Axiom+]{Property}
\newtheorem{Attention+}[Axiom+]{Attention}
\newtheorem{Question+}[Axiom+]{Question}
\newtheorem{Problem+}[Axiom+]{Problem}
\newtheorem{Consideration+}[Axiom+]{Consideration}
\newtheorem{Alert+}{Alert}
\newtheorem{Fact+}[Axiom+]{Fact}
\newtheorem{Remark+}[Axiom+]{Remark}

% ----------------------------
% commmand
% ----------------------------
% 執筆に便利なコマンド集です. 
% コマンドを追加する場合は下のスペースへ. 

% 集合の記号 (黒板文字)
\newcommand{\NN}{\mathbb{N}}
\newcommand{\ZZ}{\mathbb{Z}}
\newcommand{\QQ}{\mathbb{Q}}
\newcommand{\RR}{\mathbb{R}}
\newcommand{\CC}{\mathbb{C}}
\newcommand{\PP}{\mathbb{P}}
\newcommand{\KK}{\mathbb{K}}


% 集合の記号 (太文字)
\newcommand{\nn}{\mathbf{N}}
\newcommand{\zz}{\mathbf{Z}}
\newcommand{\qq}{\mathbf{Q}}
\newcommand{\rr}{\mathbf{R}}
\newcommand{\cc}{\mathbf{C}}
\newcommand{\pp}{\mathbf{P}}
\newcommand{\kk}{\mathbf{K}}

% 特殊な写像の記号
\newcommand{\ev}{\mathop{\mathrm{ev}}\nolimits} % 値写像
\newcommand{\pr}{\mathop{\mathrm{pr}}\nolimits} % 射影

% スクリプト体にするコマンド
%   例えば {\mcal C} のように用いる
\newcommand{\mcal}{\mathcal}

% 花文字にするコマンド 
%   例えば {\h C} のように用いる
\newcommand{\h}{\mathscr}

% ヒルベルト空間などの記号
\newcommand{\F}{\mcal{F}}
\newcommand{\X}{\mcal{X}}
\newcommand{\Y}{\mcal{Y}}
\newcommand{\Hil}{\mcal{H}}
\newcommand{\RKHS}{\Hil_{k}}
\newcommand{\Loss}{\mcal{L}_{D}}
\newcommand{\MLsp}{(\X, \Y, D, \Hil, \Loss)}

% 偏微分作用素の記号
\newcommand{\p}{\partial}

% 角カッコの記号 (内積は下にマクロがあります)
\newcommand{\lan}{\langle}
\newcommand{\ran}{\rangle}



% 圏の記号など
\newcommand{\Set}{{\bf Set}}
\newcommand{\Vect}{{\bf Vect}}
\newcommand{\FDVect}{{\bf FDVect}}
\newcommand{\Ring}{{\bf Ring}}
\newcommand{\Ab}{{\bf Ab}}
\newcommand{\Mod}{\mathop{\mathrm{Mod}}\nolimits}
\newcommand{\CGA}{{\bf CGA}}
\newcommand{\GVect}{{\bf GVect}}
\newcommand{\Lie}{{\bf Lie}}
\newcommand{\dLie}{{\bf Liec}}



% 射の集合など
\newcommand{\Map}{\mathop{\mathrm{Map}}\nolimits}
\newcommand{\Hom}{\mathop{\mathrm{Hom}}\nolimits}
\newcommand{\End}{\mathop{\mathrm{End}}\nolimits}
\newcommand{\Aut}{\mathop{\mathrm{Aut}}\nolimits}
\newcommand{\Mor}{\mathop{\mathrm{Mor}}\nolimits}

% その他便利なコマンド
\newcommand{\dip}{\displaystyle} % 本文中で数式モード
\newcommand{\e}{\varepsilon} % イプシロン
\newcommand{\dl}{\delta} % デルタ
\newcommand{\pphi}{\varphi} % ファイ
\newcommand{\ti}{\tilde} % チルダ
\newcommand{\pal}{\parallel} % 平行
\newcommand{\op}{{\rm op}} % 双対を取る記号
\newcommand{\lcm}{\mathop{\mathrm{lcm}}\nolimits} % 最小公倍数の記号
\newcommand{\Probsp}{(\Omega, \F, \P)} 
\newcommand{\argmax}{\mathop{\rm arg~max}\limits}
\newcommand{\argmin}{\mathop{\rm arg~min}\limits}





% ================================
% コマンドを追加する場合のスペース 
\newcommand{\UU}{\mcal{U}}
\newcommand{\OO}{\mcal{O}}
\newcommand{\emp}{\varnothing}
\newcommand{\ceq}{\coloneqq}
\newcommand{\sbs}{\subset}
\newcommand{\mapres}[2]{\left. #1 \right|{#2}}
\newcommand{\ded}{\hfill $\blacksquare$}

% 自前の定理環境
%   https://mathlandscape.com/latex-amsthm/
% を参考にした
\newtheoremstyle{mystyle}%   % スタイル名
    {5pt}%                   % 上部スペース
    {5pt}%                   % 下部スペース
    {}%              % 本文フォント
    {}%                  % 1行目のインデント量
    {\bfseries}%                      % 見出しフォント
    {.}%                     % 見出し後の句読点
    {12pt}%                     % 見出し後のスペース
    {\thmname{#1}\thmnumber{ #2 }\thmnote{{\normalfont (#3)}}}% % 見出しの書式

\theoremstyle{mystyle}
\newtheorem{AXM}{公理}[section]
\newtheorem{DFN}[Axiom]{定義}
\newtheorem{THM}[Axiom]{定理}
\newtheorem{PRP}[Axiom]{命題}
\newtheorem{LMM}[Axiom]{補題}
\newtheorem{CRL}[Axiom]{系}
\newtheorem{EG}[Axiom]{例}

%\newtheorem{}{Axiom}[]

% =================================





% ---------------------------
% new definition macro
% ---------------------------
% 便利なマクロ集です

% 内積のマクロ
%   例えば \inner<\pphi | \psi> のように用いる
\def\inner<#1>{\langle #1 \rangle}

% ================================
% マクロを追加する場合のスペース 

%=================================





% ----------------------------
% documenet 
% ----------------------------
% 以下, 本文の執筆スペースです. 
% Your main code must be written between 
% begin document and end document.
% ---------------------------


\begin{document}

\title{代数曲線論}
\author{Toshi2019}

\date{Feb 20, 2022}

\maketitle
\begin{abstract}
    報告者は,21年度の秋セメスターに\cite{ogs}を用いてセミナーを行った.
    本稿では,このセミナーで勉強した内容を報告する.
    まず複素関数論について復習してからリーマン球面について述べる.
    その後,本稿ではほとんど1次元の場合,すなわちリーマン面の場合
    しか扱わないが,一般の次元に対して複素多様体を
    定義する.セミナーで学んだ事実のうち興味あるものとして,
    リーマン・ロッホの定理をリーマン球面に対して適用したものと
    代数学の基本定理がある.これらの証明が本稿の目標である.
    最後に,特異点のない代数曲線について,
    方程式の解としての構造と複素多様体としての構造の間の
    対応について述べる.
\end{abstract}

\section*{凡例}
本稿では,次の記号について断りなく用いる.

\begin{itemize}
    \item $\zz,\rr,\cc$は整数,実数,複素数全体の集合を表す.
    \item 何らかの族$(x_i)_{i\in I}$について,添字集合が
    明らかな場合は$(x_i)_{i}$のように表すことがある.
    \item $X$を集合とする.たんに$X$の関数というときには,
    $X$上の複素数に値を取る写像とする.
    \item 差集合:集合$X$の元のうち$A$に属さないもの全体を$X-A$で表す.
    \item 球面:$S^{n-1}\coloneqq \{x\in \rr^{n}; \left\|x\right\|=1\}$
    \item 射影:直積集合に対し第$i$成分を対応させる写像を$\pr_i\colon\prod_{i} X_i\to X_i$とかく.
\end{itemize}
\section{複素関数論}

$f$を複素数平面全体で定義された複素数に値を取る関数とする.

\section{リーマン面}

\subsection{リーマン球面}

$\cc^{2}$から原点$0=(0,0)$を除いた集合$\cc^{2}-\{0\}$
の点$(a_{0},a_{1}), (b_{0},b_{1})$に対し次の関係を考える.
\begin{align}\label{eq:sim1}
    (a_{0},a_{1})\sim (b_{0},b_{1})
    \Longleftrightarrow
    (a_{0},a_{1})= c\cdot(b_{0},b_{1})
    \text{となる複素数}c\neq0\text{が存在する.}
\end{align}
これは同値関係である.
$(a_0,a_1)$の同値類$\{c\cdot(a_0,a_1); c\in\cc-\{0\}\}$を
$[a_0\colon a_1]$とかく.
\begin{proof}[\eqref{eq:sim1}\textbf{が同値関係になることのチェック}]
    
    (反射律)
    $c=1$は$(a_{0},a_{1})= 1\cdot(a_{0},a_{1})$をみたす.

    (対称律)
    複素数$c\neq0$を$(a_{0},a_{1})= c \cdot (b_{0},b_{1})$
    をみたすものとすると,複素数$c^{-1}\neq0$は
    $(b_{0},b_{1})=c^{-1} \cdot(a_{0},a_{1})$
    をみたす.

    (推移律)
    複素数$c,c'\neq0$をそれぞれ
    $(a_{0},a_{1})= c \cdot (b_{0},b_{1})$,
    $(b_{0},b_{1})= c' \cdot (c_{0},c_{1})$
    をみたすものとする.このとき複素数$cc'\neq0$は
    \begin{align*}
        (a_{0},a_{1})
        = c \cdot (b_{0},b_{1})
        =cc' \cdot (c_{0},c_{1})
    \end{align*}
    をみたす.
\end{proof}

同値関係${\sim}$の定める商写像を用いて次の集合を定義する.

\begin{Definition}
    $\pp^{1}(\cc) = \left(\cc^{2}-\{0\}\right)/{\sim}$
    を\textbf{リーマン球面}(Riemann sphere) という.
    $\pp^{1}(\cc)$を$\pp^{1}_{\cc}$とか$\pp^{1}$ともかく.
\end{Definition}

$\pp^1$の任意の点$P$は$[a_0\colon a_1]$の形に表せる.
実際,$P$を$\pp^1$の点とすると,$\pp^1$の
定義より,$(a_{0},a_{1})\in \cc^{2}-\{0\}$で
$P=\pi(a_{0},a_{1})=[a_0\colon a_1]$となるものが存在する.

また,$[a_0\colon a_1]=[b_0\colon b_1]$となるのは,
$a_0\colon a_1=b_0\colon b_1$となるときである.
実際,
\begin{align*}
    [a_0\colon a_1]=[b_0\colon b_1]
    &\Longleftrightarrow
    (a_0, a_1)\in[b_0\colon b_1]\\
    &\Longleftrightarrow
    (a_0, a_1)\sim(b_0, b_1)\\
    &\Longleftrightarrow
    (a_0, a_1)=c(b_0, b_1) \text{ for some }c\neq0\\
    &\Longleftrightarrow
    a_0=cb_0, a_1=cb_1 \text{ for some }c\neq0\\    
    &\Longleftrightarrow
    a_0\colon b_0 = a_1 \colon b_1\\
    &\Longleftrightarrow
    a_0 b_1 = a_1 b_0\\
    &\Longleftrightarrow
    a_0\colon b_1 = b_0 \colon b_1
\end{align*}
である.

\begin{Definition}
    次の写像の組を考える.
    $\begin{tikzcd}
      {\cc^{2}-\{0\}}
        \arrow[r, shift left ,"\pr_1=X_0"]
        \arrow[r, shift right,"\pr_2=X_1"']
      & {\cc}
    \end{tikzcd}; (a_0,a_1)\mapsto a_0,a_1.$
    この組を$\cc^{2}-\{0\}$の標準座標,$\pp^1$の同次座標という.
\end{Definition}
$P\in\pp^1$を代表する$\cc^{2}-\{0\}$の点$\tilde{P}$の
標準座標の値$(a_0,a_1)$が$P$の同次座標の値である.
なお,$P$に対する$\tilde{P}$の取り方,
すなわち$(a_0,a_1)$の取り方には
任意性がある.

$\pp^{1}$は商写像$\pi \colon \cc^{2}-\{0\}
\longrightarrow\left(\cc^{2}-\{0\}\right)/{\sim}$
による商位相により位相空間になる.この定義から$\pi$の連続性が従う.

$\pp^1$の位相空間としての性質を調べるために,次の部分集合を定義する.
\begin{align*}
    U_0=\{[a_0\colon a_1]\in\pp^1; a_0\neq0\},\\
    U_1=\{[a_0\colon a_1]\in\pp^1; a_1\neq0\}.
\end{align*}
このとき次が成り立つ.
\begin{align*}
    U_0\cup U_1 &= \pp^1,\\
    U_0\cap U_1 
    &= \{[a_0\colon a_1]\in\pp^1; a_0, a_1\neq0\}\\
    &= U_0 - \{[1\colon 0]\}\\
    &= U_1 - \{[0\colon 1]\}.
\end{align*}

\begin{Lemma}\label{mnf:p1}
    1. 
    商写像$\pi \colon \cc^{2}-\{0\}
    \longrightarrow\left(\cc^{2}-\{0\}\right)/{\sim}$
    は開写像である.

    2. 
    $U_0$と$U_1$は$\pp^1$の開集合であり,
    \begin{align*}
        \pphi_0\colon U_0\overset{{\sim}}{\longrightarrow}\cc; [a_0\colon a_1]\mapsto a_1/a_0,
        \pphi_1\colon U_1\overset{{\sim}}{\longrightarrow}\cc; [a_0\colon a_1]\mapsto a_0/a_1,
    \end{align*}
    はともに同相写像である.

    3. 
    任意の$\dip A = \begin{bmatrix}
        a&b\\c&d
    \end{bmatrix}\in GL(2,\cc)$に対し,自己同相写像
    \begin{align*}
        p_A\colon \pp^1\overset{{\sim}}{\longrightarrow}\pp^1;
        \begin{bmatrix}
            a_0\\a_1
        \end{bmatrix}
        \mapsto
        \begin{bmatrix}
            a&b\\c&d
        \end{bmatrix}
        \begin{bmatrix}
            a_0\\a_1
        \end{bmatrix}
    \end{align*}
    が存在する.

    4. 
    $\pp^1$は第2可算公理をみたす連結なコンパクトハウスドルフ空間である.
\end{Lemma}

\begin{proof}[\textbf{証明}]
    1. 
    $U$を$\cc^2-\{0\}$の開集合とする.$\pi(U)$が$\pp^1$の
    開集合であること,すなわち$\pi^{-1}(\pi(U))$が$\cc^2-\{0\}$の
    開集合であることを示す.
    いま,任意の開集合$U\subset\cc^2-\{0\}$に対し,
    複素数$c\neq0$を用いて
    \begin{align*}
        cU = \left\{(ca_0,ca_1); (a_0,a_1)\in\cc^2-\{0\}\right\}
    \end{align*}
    とおくと,$cU$は$\cc^2-\{0\}$の開集合であり,
    \begin{align}\label{eq:proj}
        \pi^{-1}(\pi(U)) = \bigcup_{c\in\cc-\{0\}} cU \tag{$\ast$}
    \end{align}
    なので,$\pi^{-1}(\pi(U))$は$\cc^2-\{0\}$の
    開集合である.
\end{proof}

\begin{Attention}[{\eqref{eq:proj}}について]
    この等式については次の図\ref{fig:prj1}を見ると理解しやすい.
    \begin{figure}[h]
        \centering
        \begin{tikzpicture}
            \draw[>=stealth,semithick] (-2,0)--(3,0); %x軸
            \draw[>=stealth,semithick] (0,-2)--(0,4); %y軸
            \draw (0,0)node[below right]{O}; %原点
            \draw[thick, domain=-2:3] plot(\x,{0.5*\x});
            \draw[thick, domain=-1.:2] plot(\x,2*\x);
            \draw[line width=1pt] (1.5,3) rectangle (2.5,1.25); %四角
            \draw (2.5,1.25)node[below]{$2U$}; %点(2.5,1.25)
            \draw[line width=1pt] (0.75,1.5) rectangle (1.25,0.625); %四角
            \draw (0.75,1.5)node[left]{$U$}; %点(2.5,1.25)\draw[line width=1pt] (0.75,1.5) rectangle (1.25,0.625); %四角
            \draw[line width=1pt] (-0.75,-1.5) rectangle (-1.25,-0.625); %四角
            \draw (-0.75,-1.5)node[right]{$-U$}; %点(2.5,1.25)
            \draw[line width=1pt] (0.375,0.75) rectangle (0.625,0.3125); %四角
            \draw (0.625,0.2)node[right]{$(1/2)U$}; %点(2.5,1.25)
            \draw[very thick, domain=-2.:3] plot(\x,{0.8*\x});
            \draw (1.8,3.6)node[right]{$U$を通る直線の上端};
            \draw (3,2.4)node[right]{$U$を通る直線};
            \draw (3,1.5)node[right]{$U$を通る直線の下端};
        \end{tikzpicture}
        \caption{商写像の逆像1}
        \label{fig:prj1}
    \end{figure}
    いま,簡単のため$\rr^2$の部分集合$U$について考える.
    $\pi(U)$は$U$を通る直線たちの集合である.像が$\pi(U)$となる
    ようなもの,つまり$\pi^{-1}\left(\pi(U)\right)$は$U$を$c$倍に
    拡大・縮小したもの全体である.したがって\eqref{eq:proj}が成り立つ.

    次のような場合も見ておくと複素トーラスの導入のときなどに役立つ.
    \begin{align*}
        \pi\colon \rr \twoheadrightarrow \rr/\zz; x\mapsto x\mod{\zz}
    \end{align*}
    を考える.これは図\ref{fig:prj2}の上の直線を丸めて$1$次元トーラス
    にしたものである.

    \begin{figure}[h]
        \centering
        \begin{tikzpicture}[x=1cm]
            \fill[black!40] (0.7,0.1) rectangle (1.3,-0.1);
            \draw (1.3,-0.1) node[below]{$U$};
            \draw[->,>=stealth,semithick] (-5,0)--(5,0)node[below]{$\rr$}; %x軸
            \foreach \x in {-5, -4,...,4} \draw (\x,-0.2)--(\x,0.2);
            
            \draw[->,>=stealth,semithick] (0,-1)--(0,-2); %射影

            \draw[black!40,line width=5pt] (0,-5) arc (270:330: 3 and 1);
            \draw (0,-1.5) node[right]{$\pi$};
            \draw (0,-4) ellipse (3 and 1);
            \draw (1,-5.2)--(1,-4.8);
            \draw (1.3,-5.2) node[right]{$\pi(U)$};
            \draw[->,>=stealth,semithick] (0,-5.5)--(0,-6.5); %射影
            \draw (0,-6) node[right]{$\pi(U)$に来るものを集める};

            \foreach \x in {-4,-3,...,4} \fill[black!40] ({\x-0.3},-6.9) rectangle ({\x+0.3},-7.1);
            \draw (1.3,-7.1) node[below]{$U$};
            \draw (2.3,-7.1) node[below]{$U+1$};
            \draw (3.3,-7.1) node[below]{$U+2$};
            \draw (4.3,-7.1) node[below]{$U+3$};
            \draw (0.3,-7.1) node[below]{$U-1$};
            \draw (-1.3,-7.1) node[below]{$U-2$};
            \draw (-2.3,-7.1) node[below]{$U-3$};
            \draw (-3.3,-7.1) node[below]{$U-4$};
            \draw (-4.3,-7.1) node[below]{$U-5$};

            \draw[->,>=stealth,semithick] (-5,-7)--(5,-7)node[below]{$\rr$}; %x軸
            \foreach \x in {-5, -4,...,4} \draw (\x,-6.8)--(\x,-7.2);
        \end{tikzpicture}
        \caption{商写像の逆像2}
        \label{fig:prj2}
    \end{figure}
    この場合は次のようになる.
    \begin{align*}
        \pi^{-1}\left(\pi(U)\right) 
        &= (U\text{を射影したものと同じところに行くもの})\\
        &= (U\text{を整数分だけずらしたもの全て})\\
        &= \bigcup_{n\in\zz}(U+n).
    \end{align*}
\end{Attention}











\clearpage


$\cc^{n}$での座標が$z=(z_{1},\dots,z_{n})$であるとき
複素数空間$\cc^{n}$を
$\cc^{n}_{z}$とか$\cc^{n}_{(z_{1},\dots,z_{n})}$とかく.

\begin{Definition}[$n$次元複素多様体]
    $X$を位相空間とする.
    $(\pphi_{i}\colon U_{i}\to \UU_{i})_{i\in I}$を
    写像の族とする.このとき,対$(X,(\pphi_{i})_{i})$が
    次の条件(1)--(4)をみたすとき,$X$を台集合とし$(\pphi_{i})_i$を
    座標近傍系とする\textbf{$n$次元複素多様体} ($n$-dimensional complex manifold) という.

    (1) 
    $X$は空集合でなく,第2可算公理を満たす連結なハウスドルフ空間である.

    (2) 
    すべての$i\in I$に対して$U_i$は$X$の空でない開集合であり,
    $(U_i)_i$は$X$の開披覆である.

    (3) 
    すべての$i\in I$に対して,
    $\UU_i$は$\cc^{n}_{(z_{1},\dots,z_{n})}$の
    空でない開集合であり$\pphi_{i}\colon U_i\to\UU_i$は同相である.

    (4) 
    任意の$i\neq j \in I$で$U_i\cap U_j \neq \emp$をみたすもの
    に対して$\UU_{ij}\ceq \pphi_j(U_i\cap U_j)\sbs \UU_j$と
    おくとき,$\pphi_{ij}\ceq 
    \mapres{\pphi_{i}\circ\pphi_{j}^{-1}}{\UU_{ij}}
    \colon 
    \UU_{ij}\to\UU_{ji}$は正則である.
\end{Definition}

1次元複素多様体を\textbf{リーマン面} (Riemann surface) という.

\section{1次元における諸結果}

% $\pp^{1}$の定義
% 有理形関数 と 極の位数

次はリーマンロッホの定理の特殊な場合である.

\begin{Theorem}[{\cite[命題1.14]{ogs}}]
    $P_{1},\dots,P_{m}$を$\pp^{1}$の互いに相異なる点とする.
    $\pp^{1}$上の有理形関数で各$P_i$のみで高々$n_i$の極をもつもの全体
    \begin{align*}
        \Gamma\left(\pp^{1}, \OO_{\pp^{1}}\left(\sum_{i=1}^{m}n_{i}P_{i}\right)\right)
    \end{align*}
    は関数の和とスカラー倍によって$\cc$ベクトル空間を成す.
    さらにその空間の次元は次で与えられる.
    \begin{align*}
        \dim\Gamma\left(
            \pp^{1}, \OO_{\pp^{1}}\left(
                \sum_{i=1}^{m}n_{i}P_{i}
                \right)
            \right)
        = 1+\sum_{i=1}^{m}n_{i}.
    \end{align*}
\end{Theorem}


\section{コンパクトリーマン面}

\begin{Theorem}[代数学の基本定理{\cite[系2.24]{ogs}}]
    $n\geqq1$とする.複素数を係数とする$n$次方程式
    \begin{align*}
        a_{0}z^{n}+a_{1}z^{n-1}+\dots +a_{n}=0,\quad a_{0}\neq 0
    \end{align*}
    に対し,複素数解$\alpha$が存在する.
\end{Theorem}

\section{それから}

\begin{thebibliography}{15}

    \bibitem{ueno} 上野健爾, 代数幾何, 岩波書店, 2005.

    \bibitem{umemura} 梅村浩, 楕円関数論 楕円曲線の解析学 [増補新装版], 東京大学出版会, 2020.

    \bibitem{ogs} 小木曽啓示, 代数曲線論, 朝倉書店, 2002.

    \bibitem{kaneko} 金子晃, 関数論講義 (ライブラリ数理・情報系の数学講義, 5), サイエンス社, 2021.

    \bibitem{kobayashi1} 小林昭七, 複素幾何1 (岩波講座現代数学の基礎, 29), 岩波書店, 1997.

    \bibitem{kobayashi2} 小林昭七, 複素幾何2 (岩波講座現代数学の基礎, 30), 岩波書店, 1998.

    \bibitem{jimbo} 神保道夫, 複素関数論 (岩波講座現代数学への入門, 4), 岩波書店, 1995.

    \bibitem{takebe} 武部尚志, 楕円積分と楕円関数 おとぎの国の歩き方, 日本評論社, 2019.

    \bibitem{fuji} 藤本坦孝, 複素解析 (岩波講座現代数学の基礎, 3), 岩波書店, 1996.

    \bibitem{yoshida} 吉田洋一, 函数論 第2版 (岩波全書, 141), 岩波書店, 1973.

    \bibitem{tmp} 著者, 書名 (シリーズ名, 巻数), 発行所, 発行都市名, 年号
\end{thebibliography}

\end{document}


