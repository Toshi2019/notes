%==========================================
%   Residues and duality TeX project
%   2023.Jun.22 - 
%   written by Toshihiro Oshiba
%==========================================

\documentclass[leqno]{book}


%usepackage
%------------------------
\usepackage{amsmath}
\usepackage{amsthm}
\usepackage[psamsfonts]{amssymb}
\usepackage{color}
\usepackage{ascmac}
\usepackage{amsfonts}
\usepackage{mathrsfs}
%\usepackage{rsfso}
\usepackage{mathtools}
\usepackage{amssymb}
\usepackage{graphicx}
\usepackage{fancybox}
\usepackage{enumerate}
\usepackage{verbatim}
\usepackage{subfigure}
\usepackage{proof}
\usepackage{listings}
\usepackage{otf}
\usepackage{algorithm}
\usepackage{algorithmic}
\usepackage{tikz}
\usepackage[all]{xy}
\usepackage{amscd}

\usepackage[pdfborder=0,dvipdfmx]{hyperref}
\usepackage{xcolor}
\definecolor{darkgreen}{rgb}{0,0.45,0} 
\definecolor{darkred}{rgb}{0.75,0,0}
\definecolor{darkblue}{rgb}{0,0,0.6} 
\hypersetup{
    colorlinks=true,
    citecolor=darkgreen,
    linkcolor=darkblue,
    urlcolor=darkblue,
}
\usepackage{pxjahyper}
\usepackage{comment}
\usepackage{enumitem}
\usepackage{layout}

\usetikzlibrary{cd}

%\theoremstyle{definition}
\theoremstyle{plain}
\newtheorem{thm}{Theorem}[section]

\theoremstyle{definition}
\newtheorem{dfn}[thm]{Definition}
\newtheorem{lem}[thm]{Lemma}

\theoremstyle{remark}
\newtheorem{rem}[thm]{Remark}

\renewcommand{\qedsymbol}{q.e.d.}

\newcommand{\ZZ}{\mathbb{Z}}
\newcommand{\RR}{\mathbb{R}}
\newcommand{\CC}{\mathbb{C}}

% category theory
\newcommand{\fin}[1]{{#1}^{\mathrm{f}}} % finiteness


% sheaf theory
\newcommand{\F}{\mathscr{F}}
\newcommand{\HH}{\mathscr{H}}
\newcommand{\shhol}[1]{\mathscr{O}_{#1}}
\newcommand{\shan}[1]{\mathscr{A}_{#1}}
\newcommand{\shor}[1]{\omega_{#1}}
\newcommand{\shhyp}[1]{\mathscr{B}_{#1}}

% D-modules
\newcommand{\DD}{\mathscr{D}}
\newcommand{\PP}{\mathscr{P}}


%new definition macro
%-------------------------
\def\inner<#1>{\langle #1 \rangle}

\newcommand{\mapres}[2]{\left. #1 \right|_{#2}}

\numberwithin{equation}{section}


%==============================================================
% page layout 
%--------------------------------------------------------------

%\setlength{\oddsidemargin}{-1cm}
\setlength{\evensidemargin}{\oddsidemargin}
\pagestyle{myheadings}






%==============================================================






\title{Residues and duality}
\author{}

\begin{document}
\maketitle

\frontmatter

\tableofcontents
\markboth{Contents}{Contents}

%\chapter*{Preface for Part II}

This is the last of two parts of the Proceedings of
the conference on Hyperfunctions and Pseudo-Differential 
Equations held at Katata on October 12--14, 1971. 

This part consists of a paper by 
M. Sato, T. Kawai and M. Kashiwara which is an enlarged 
version of four lectures by them delivered at the conference.

We received the final manuscript in June, 1971 but have
postponed the publication because the authors had the intention
of adding an introduction to the paper. 
Since we do not think it appropriate to wait for it forever, 
we have decided to publish this part in the present form. 

In place of the introduction, we advise the reader to read 
the lectures by the authors at different occasions, 
the Nice Congress, 1970, the A. M. S. Symposium on 
Partial Differential Equations at Berkeley, 1971, 
and the Colloque C. N. R. S. Equations aux 
D\'eriv\'ees Partielles Lin\'eaires at Orsay, 1972. 

We thank Miss C. Sagawa for typing and Mr.\ T. Miwa and 
Mr.\ T.\ Oshima for proof-reading. 


\begin{flushright}
    December 28, 1972
    \linebreak

    Hikosaburo Komatsu
\end{flushright}

\layout
\mainmatter

%\chapter{Theory of Microfunctions}

\section{Construction of the sheaf of microfunctions}

\subsection{Hyperfunctions}

Let $M$ be an $n$-dimentional real analytic manifold 
and $X$ be a complex neifgborhood of $M$. 




\subsection{Real monoidal transformation and real comonoidal transformation}

Now consider the following situation, although we apply it 
to a special case in this section.




\subsection{Definition of microfunctions}

Now we will come back to the original situation. 




\subsection{Sheaves on sphere bundle and on cosphere bundle}

We consider the following situation.



\subsection{Fundamental diagram on $\mathscr{C}$}

We will apply the arguements in the preceding section 
to a special case.

\section{Several oprations on hyperfunctions and microfunctions}

\chapter{Introduction}

The main purpose of these notes is to prove a
duality theorem for cohomology of quasi-coherent sheaves, 
with respect to a proper morphism of locally noetherian 
preschemes. 
Various such theorems are already known. 
Typical is the duality theorem for a non-singular complete 
curve $X$ over an algebraically closed field $k$, 
which says that
\begin{equation*}
    h^0(D)=h^1(K-D),
\end{equation*}
where $D$ is a divisor, $K$ is the canonical divisor, and
\begin{equation*}
    h^i(D)=\dim_K H^i(X,L(D))
\end{equation*}
for any i, and any divisor D. 
(See e.g. \cite[Ch.~II]{SerreGr} for a proof.)

Various attempts were made to generalize this theorem 
to varieties of higher dimension, 
and as Zariski points out in his report [20], 
his generalization of a lemma of 
Enriques-Severi [19] is equivalent to the statement that







%\section{Several oprations on hyperfunctions and microfunctions}

\subsection{Linear differential operators}

\subsection{Substitution}

\subsection{Integration along fibers}

\subsection{Products}

\subsection{Micro-local operators}

\subsection{Complex conjugation}
%\section{Techniques for construction of hyperfunctions and microfunctions}

\subsection{Real analytic functions of positive type}

\subsection{Boundary values of hyperfunctions with holomorphic parameters and examples}
%\chapter{Foundation of the Theory of Pseudo-differential Equations}

\section{Definition of pseudo-differential operators}

Is a


%\section{Fundamental properties of pseudo-differential operators}

\subsection{Theorems on ellipticity and the equivalence of pseudo-differential operators}

\subsection{Theorems on division of pseudo-differential operators}
%\section{Algebraic properties of the sheaf of pseudo-differential operators}


\subsection{Pseudo-differential operators with holomorphic parameters}


\subsection{Properties of the ring of formal pseudo-differential operators}


\subsection{Contact structure and quantized contact transforms}


\subsection{Faithful flatness}


\begin{rem}
    Let $X$ be a complex manifold. We denote by $\DD_X$ (resp. $\fin{\DD}_X$) 
    the sheaf of differential operators on $X$ (resp. differential 
    operators of finite order on $X$). 
    $\PP_X$ (resp. $\fin{\PP}_X$) is a $\pi^{-1}\DD_X$-Algebra 
    (resp. $\pi^{-1}\fin{\DD}_X$-Algebra). 
    By using the method 
\end{rem}

\subsection{Operations on systems of pseudo-differential equations}


%\section{Maximally overdetermined systems}

\subsection{Definition of maximally overdetermined systems}

\subsection{Invariants of maximally overdetermined systems}

\subsection{Quantized contact transform --- general case ---}
%\section{Structure theorem for systems of pseudo-differential equations in the complex domain}
\markright{\scriptsize{2.5 Structure theorem for systems of pseudo-differential equations in the complex domain}}
In this section we establish the fundamental theorem 
concerning the structure of a system of pseudo-differential equations 
of finite order in complex domain at generic points, 
i.e., we will firstly prove in theorem \ref{thm512} 
as the simplest case that any system $\mathscr{M}$ of 
pseudo-differential equations of finite order with one unknown
function and simple characteristics can be transformed micro-locally 
into the partial de Rham systems
\begin{align*}
    \mathscr{N} \colon \frac{\partial}{\partial x'_i}u = 0, \quad i = 1,\ldots,d
\end{align*}
by a suitable ``quantized'' contact transformation.
Here ``micro-locally'' means ``locally on $P^\ast X$, not on $X$''. 
In the sequal we use the word ``micro-locally'' 
in this sense (and sometimes in the sense that ``locally on 
$S^{\ast}M$, not on $M$'' when we consider the problems in the real domain).
Later we extend Theorem\ref{thm512} to more general systems 
by the aid of pseudo-differential operators of infinite order.

\subsection{Structure theorem for systems of pseudo-differential equations with simple characteristics}

\begin{thm}\label{thm512}
    a
    % theorem5.1.2
\end{thm}

\subsection{Equivalence of pseudo-differential operators with constant multiple characteristics}

\begin{rem}
    This example shows that the structure of the hyperfunction
    solution sheaf, not merely the microfunction solution sheaf, of 
    the equation of $P_1(D)u = 0$ and that of $P_2(D)u = 0$ are the same, 
    because the operators $A_j(x,D)$ are differential operators, not merely 
    pseudo-differential operators. Note that, more generally, 
    if $P(x,D)$ is a linear differential operator of order $m$ 
    defined in a neighborfood of the origin of $\CC^n$ whose 
    principal symbol is $\iota^m_1$, then the differential equation 
    $P(x,D)u=0$ and $D_1^mu=0$ are equivalent as left $\DD$-modules.
\end{rem}
\subsection{Structure theorem for regular systems of pseudo-differential equations}

%\chapter{Structure of Systems of Pseudo-differential Equations}

\section{Realification of holomorphic microfunctions}

\subsection{Realification of holomorphic hyperfunctions}

\subsection{Realification of holomorphic microfunctions}

\subsection{Real ``quantized'' contact transforms}

%\section{Structure theorems for systems of pseudo-differential equations in the real domain}

\subsection{Structure theorem I --- partial de Rham type ---}

\subsection{Structure theorem II --- partial Cauchy Riemann type ---}

\subsection{Structure theorem III --- Lewy-Mizohata type ---}

\subsection{Structure theorem IV --- general case ---}


\backmatter

%\begin{thebibliography}{99}
    \par
      \bibitem[Liu]{Liu} Qing Liu, 
      \textit{Algebraic Geometry and Arithmetic Curves}, 
      Oxford Graduate Text in Mathematics, 
      \textbf{6}, 2010.
\end{thebibliography}
\begin{thebibliography}{20}
    \par
    \bibitem[Hatshorne1]{Hartshorne1} R. Hartshorne, 
    \textit{Residues and duality}, Lecture Notes in Mathematics, 
    Vol.\ 20, Springer-Verlag, Berlin, 1966.

    \bibitem[Sato1]{Sato1} Mikio Sato, \textit{Theory of hyperfunctions II}, J. Fac.\ Sci.\ Univ.\ Tokyo, {\bf{8}} (1960), 387--437.

    \bibitem[16]{SerreGr} Jeqn-Pierre Serre, 
    \textit{Groupes alg\'ebriques et corps de classes}, Paris, Herman (1959).

    \end{thebibliography}
\end{document}