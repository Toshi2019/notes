% -----------------------
% preamble
% -----------------------
% Don't change preamble code yourself. If you add something(usepackage, newtheorem, newcommand, renewcommand),
% please tell them editor of institutional paper of RUMS.

%documentclass
%------------------------
\documentclass[11pt, a4paper, dvipdfmx]{jsarticle}


%usepackage
%------------------------
\usepackage{amsmath}
\usepackage{amsthm}
\usepackage[psamsfonts]{amssymb}
\usepackage{color}
\usepackage{ascmac}
\usepackage{amsfonts}
\usepackage{mathrsfs}
\usepackage{amssymb}
\usepackage{graphicx}
\usepackage{fancybox}
\usepackage{enumerate}
\usepackage{verbatim}
\usepackage{subfigure}
\usepackage{proof}
\usepackage{listings}
\usepackage{otf}
\usepackage{algorithm}
\usepackage{algorithmic}
\usepackage{tikz}
\usepackage[all]{xy}
\usepackage{amscd}

\usepackage[dvipdfmx]{hyperref}
\usepackage{pxjahyper}

\usetikzlibrary{cd}

%theoremstyle
%--------------------------
\theoremstyle{definition}


%newtheoem
%--------------------------
%If you want to use theorem environment in Japanece, You can use these code. 
%Attention
%--------------------------
%all theorem enivironment number depend on only section number.
\newtheorem{Axiom}{公理}[section]
\newtheorem{Definition}[Axiom]{定義}
\newtheorem{Theorem}[Axiom]{定理}
\newtheorem{Proposition}[Axiom]{命題}
\newtheorem{Lemma}[Axiom]{補題}
\newtheorem{Corollary}[Axiom]{系}
\newtheorem{Example}[Axiom]{例}
\newtheorem{Claim}[Axiom]{主張}
\newtheorem{Property}[Axiom]{性質}
\newtheorem{Attention}[Axiom]{注意}
\newtheorem{Question}[Axiom]{問}
\newtheorem{Problem}[Axiom]{問題}
\newtheorem{Consideration}[Axiom]{考察}
\newtheorem{Alert}[Axiom]{警告}


%----------------------------
%If you want to use theorem environment with no number in Japanese, You can use these code.
\newtheorem*{Axiom*}{公理}
\newtheorem*{Definition*}{定義}
\newtheorem*{Theorem*}{定理}
\newtheorem*{Proposition*}{命題}
\newtheorem*{Lemma*}{補題}
\newtheorem*{Example*}{例}
\newtheorem*{Corollary*}{系}
\newtheorem*{Claim*}{主張}
\newtheorem*{Property*}{性質}
\newtheorem*{Attention*}{注意}
\newtheorem*{Question*}{問}
\newtheorem*{Problem*}{問題}
\newtheorem*{Consideration*}{考察}
\newtheorem*{Alert*}{警告}

%--------------------------
%If you want to use theorem environment in English, You can use these code.
%--------------------------
%all theorem enivironment number depend on only section number.
\newtheorem{Axiom+}{Axiom}[section]
\newtheorem{Definition+}[Axiom+]{Definition}
\newtheorem{Theorem+}[Axiom+]{Theorem}
\newtheorem{Proposition+}[Axiom+]{Proposition}
\newtheorem{Lemma+}[Axiom+]{Lemma}
\newtheorem{Example+}[Axiom+]{Example}
\newtheorem{Corollary+}[Axiom+]{Corollary}
\newtheorem{Claim+}[Axiom+]{Claim}
\newtheorem{Property+}[Axiom+]{Property}
\newtheorem{Attention+}[Axiom+]{Attention}
\newtheorem{Question+}[Axiom+]{Question}
\newtheorem{Problem+}[Axiom+]{Problem}
\newtheorem{Consideration+}[Axiom+]{Consideration}
\newtheorem{Alert+}{Alert}

%commmand
%----------------------------
\newcommand{\N}{\mathbb{N}}
\newcommand{\Z}{\mathbb{Z}}
\newcommand{\Q}{\mathbb{Q}}
\newcommand{\R}{\mathbb{R}}
\newcommand{\C}{\mathbb{C}}
\newcommand{\F}{\mathcal{F}}
\newcommand{\X}{\mathcal{X}}
\newcommand{\Y}{\mathcal{Y}}
\newcommand{\Hil}{\mathcal{H}}
\newcommand{\RKHS}{\Hil_{k}}
\newcommand{\Loss}{\mathcal{L}_{D}}
\newcommand{\MLsp}{(\X, \Y, D, \Hil, \Loss)}
\newcommand{\p}{\partial}
\newcommand{\h}{\mathscr}
\newcommand{\mcal}{\mathcal}
\newcommand{\lan}{\langle}
\newcommand{\ran}{\rangle}
\newcommand{\pal}{\parallel}
\newcommand{\dip}{\displaystyle}
\newcommand{\e}{\varepsilon}
\newcommand{\dl}{\delta}
\newcommand{\pphi}{\varphi}
\newcommand{\ti}{\tilde}

\renewcommand{\P}{\mathbb{P}}
\newcommand{\Probsp}{(\Omega, \F, \P)}

%new definition macro
%-------------------------
\def\inner<#1>{\langle #1 \rangle}

\usepackage[stable]{footmisc}
%\renewcommand{\proofname}{\bf 証明} % 「証明」の見出しを日本語にする
\newcommand{\pr}{\mathop{\mathrm{pr}}\nolimits} % 射影の記号を斜字体にしない
\numberwithin{equation}{section} % 式番号を「(3.5)」のように印刷
\def\rnum#1{\expandafter{\romannumeral #1}} 
\def\Rnum#1{\uppercase\expandafter{\romannumeral #1}} 

%----------------------------
%documenet 
%----------------------------
% Your main code must be written between begin document and end document.
\title{実数論}
\author{Toshi2019}
\date{2021-04-18}
\begin{document}
\maketitle

\begin{abstract}
\end{abstract}
\section{凡例}

よく用いる用語と記号をまとめておく.
\begin{itemize}
  \item 数列: 
  数列 $(a_n)_{n\in\N} = (a_0,a_1,\dots)$ を 
  $(a_n)$ のように略記することがある.
  \item 開球: 
  ${\R}^n$をユークリッド距離による距離空間と考え, 
  $a\in{\R}^n$を中心とする半径$r > 0$の開球を
  $U_r(a) = \{x\in{\R}^n\mid d(a,x)\le r\}$
  で表す. 
  \item 開集合: 
  $X$ を位相空間とする. $U$ が $X$ の開集合であることを
  $U\underset{\text{open}}{\subset}X$ のようにかくことがある. 
\end{itemize}

\section{導入}

$1/3 = 0.333\cdots$であることはよく知られた事実である. 中学校では
$x = 0.333\cdots$とおき, 両辺に10を掛けて$10x = 3.333\cdots$. 
辺々引いて$9x = 3$. 最後に両辺を9で割って$0.333\cdots = x = 1/3$
のようにすると習った. 

これを示すのに, 高校では極限を教わった上で
\begin{align*}
  0.333\cdots 
  &= 0.3 + 0.03 + 0.003 + \cdots\\
  &= \lim_{n\to\infty}\sum_{k=1}^{n}\frac{3}{10^k}\\
  &= \lim_{n\to\infty}\left(\frac{3}{10}
  \frac{1-(1/10)^n}{1-1/10} \right)\\
  &= \lim_{n\to\infty}\frac{1}{3}
  \left(1-\left(\frac{1}{10}\right)^n\right)\\
  &= \frac{1}{3}\times(1-0) = \frac{1}{3}
\end{align*}
と習った. 

大学に入ってすぐの微積分の授業で極限を厳密に定義することで, 
これがきちんと基礎付けられた. すなわち, 任意の自然数$n\geqq1$に対し, 
\begin{align*}
  \left|\sum_{k=1}^{n}\frac{3}{10^k} - \frac{1}{3}\right| 
  &= \left|\frac{1}{3}
  \left(1-\left(\frac{1}{10}\right)^n\right)
  - \frac{1}{3}\right|\\
  &= \left|-\left(\frac{1}{10}\right)^n\right|\\
  &= \left(\frac{1}{10}\right)^n < \frac{1}{n}
\end{align*}
が成り立つので, 後述する
はさみうちの原理(命題\ref{Pr:sq})より, 
$0.333\dots=1/3$ということになるのであった. 

このように, 無限小数を, 無限級数と捉えることで, 
特に有理数の場合には, 容易に極限を求められる. 

\section{実数の連続性}

次を公理とする. 

\begin{Axiom}\label{Ax:RNum}
  
  \quad 1. 
  ({\bf アルキメデスの原理}(axiom of Archimedes)) 
  $a$ が実数ならば, $n \leqq a \leqq n + 1$ 
  をみたす整数$n$が存在する. 
    
  2. 
  $(a_n)$ を自然数列で, 任意の $n \geqq 0$ に対し, 
  $a_n = 0$ か $a_n = 1$ のどちらかであるものとする. 
  このとき, 実数$b$で, 全ての自然数$m \geqq 0$に対し
  \begin{eqnarray}\label{axEq}
    \sum_{n=1}^{m}\frac{a_n}{2^n} 
    \leqq b 
    \leqq \sum_{n=1}^{m}\frac{a_n}{2^n} + \frac{1}{2^m}
  \end{eqnarray}
  をみたすものが存在する. 
\end{Axiom}
% Ax1.1.1.2

$m = 0$ とすると, 
$\dip \sum_{n=1}^{m}\frac{a_n}{2^n} = 0$
となり不等式\eqref{axEq}は
$0 \leqq b \leqq 1$
を表す. 

公理\ref{Ax:RNum}から, 実数は閉区間$\left[0,1\right]$
を整数で左右にずらしたもので覆うことができ, それらを整数部分
と2進小数で表せる. 次の命題の系から式\eqref{axEq}の $b$ は
ただ1つであることがわかる. 

\begin{Proposition}[有理数の稠密性]\label{Pr:dnsQ}
  $a$と$b$を実数とする. $a<b$ならば, 
  有理数$r$ で$a<r<b$をみたすものが存在する. 
\end{Proposition}

\begin{proof}[{\bf 証明}]
  $a<b$ より $b-a>0$. 自然数$n$を, 
  $\dip n = \left[\frac{1}{b-a}\right] + 1$とおく. 
  $n$ は$\dip 0 < \frac{1}{b-a} <n$
  をみたす最小の自然数である. さらに$m = [na]+1$ とおくと
  $na< m \leqq na+1 <nb$ なので, 有理数$r = m/n$ は
  $a < r < b$ をみたす. 
\end{proof}

\begin{Proposition}[はさみうちの原理(squeeze theorem)]\label{Pr:sq}
  $a$と$b$を実数とする. 任意の自然数$n\geqq1$に対し
  $\dip|a-b|<\frac{1}{n}$ならば, $a=b$ である. 
\end{Proposition}

\begin{proof}[{\bf 証明}]
  $|a-b|>0$ だったとする. 
  このとき, $\dip\frac{1}{|a-b|}>0$ は実数なので, 
  アルキメデスの公理(公理\ref{Ax:RNum}.1)より
  $\dip \frac{1}{\left|a-b\right|}\leqq n$ 
  をみたす自然数$n>0 \Leftrightarrow n\geqq1$
  が存在する. この不等式の逆数をとると, 
  $\dip \frac{1}{n}\leqq |a-b|$
  となるが, 仮定より$\dip|a-b| < \frac{1}{n}$
  なので
  $\dip \frac{1}{n}\leqq|a-b| < \frac{1}{n}$
  となりムジュン. したがって$|a-b|=0$であり, $a=b$.
\end{proof}

\begin{Corollary}
  $(a_n)$を公理{\ref{Ax:RNum}.2} の仮定をみたす数列とする. 
  不等式\eqref{axEq}をみたす実数$b$はただ1つである. 
\end{Corollary}

\begin{proof}[{\bf 証明}]
  $b$と$c$を不等式\eqref{axEq}をみたす実数とすると, 
  すべての自然数$m\geqq 1$ に対し, 
  $\dip
  \sum_{n=1}^{m}\frac{a_n}{2^n} 
  \leqq b 
  \leqq \sum_{n=1}^{m}\frac{a_n}{2^n} + \frac{1}{2^m}, 
  \quad
  \sum_{n=1}^{m}\frac{a_n}{2^n} 
  \leqq c 
  \leqq \sum_{n=1}^{m}\frac{a_n}{2^n} + \frac{1}{2^m}$
  が成り立つので, ふたつめの不等式を$-1$倍したものを
  ひとつめの不等式に辺々足して, 
  $\dip -\frac{1}{2^m} \leqq b - c \leqq \frac{1}{2^m}$
  すなわち 
  $\dip |b - c| \leqq \frac{1}{2^m}$ を得る. 
  $\dip |b - c| \leqq \frac{1}{2^m} < \frac{1}{m}$
  なので, はさみうちの原理(命題\ref{Pr:sq}) より $b=c$ である. 
\end{proof}

これで公理{\ref{Ax:RNum}.2}の$b$を特徴づけることができた. 
この$b$を
$\dip
  \sum_{n=1}^{\infty}\frac{a_n}{2^n}
$
で表す. ($\infty$の記号単体では意味を持たせていない.)
$0.a_1a_2a_3a_4\dots$ のように表し, 
$b$ の2{\bf 進小数表示}(binary decimal representation)
ということもある. (たとえば $1/2=0.1000\dots, 
5/8=1/2+1/8=0.101000\dots$のように.)


実数はいくつかの同値な性質により, その性質を様々な視点から
捉えることができる. 以下, 公理\ref{Ax:RNum}から出発して, 
それらの性質を順次示していくことにする. 

\begin{Theorem}[実数の連続性(continuity of real numbers)]\label{conOfR1}
    $a\leqq b$ を実数とし, 閉区間 $[a,b]$ の部分集合 $A$ が
    次の条件(D)をみたすとする. 
    \begin{quote}\label{aa}
        (D)\quad $x$ が $A$ の元ならば, 閉区間$[a,x]$は$A$に含まれる.
    \end{quote}
    このとき, 実数$c$で$A=[a,c]$か$A = [a,c)$の
    どちらか一方が成り立つものが存在する. 
\end{Theorem}

定理\ref{conOfR1}の実数$c$を$A$の{\bf 終点}(end point) 
と呼ぶことにする. 

\begin{proof}[{\bf 証明}]
  まず, $[a,b]= [0,1]$の場合に示し, その後$a, b$が一般の場合を
  $a=0,b=1$の場合に帰着させて示す. $a=0, b=1$とする. $0\notin A$
  のとき, 条件(D)より, $A = \emptyset$. $1\in A$ のとき, 
  条件(D)より, $A=[0,1]$. よって$0\in A, 1 \notin A$のときを考える. 

  数列$(a_n)$を次のように帰納的に定義する. $a_0=0$とする. 
  $a_n$ が自然数$m$に対し$a_m$まで定まっているとき, 
  $\dip s_m = \sum_{n=1}^{m}\frac{a_n}{2^n}$ とおく. 
  $s_0 = 0$である. $\dip s_m + \frac{1}{2^{m+1}}\in A$
  のとき, $a_{m+1}=1$とおき, そうでないときは$a_{m+1}=0$とおく. 
  
  $(a_n)$の定義と, 自然数$m$に関する帰納法により, 
  どの番号$m=0,1,2,\dots$に対しても, 
  $s_m\in A$かつ$\dip s_m+\frac{1}{2^m}\notin A$
  が成り立つことを示す.
  (\Rnum{1}) $m=0$のとき, 
  $s_0=a_0=0\in A$かつ$\dip s_0+\frac{1}{2^0}=1\notin A$. 
  ($0\in A, 1\notin A$と仮定したのだった.)
  (\Rnum{2}) $m=l$のとき, 
  $s_l\in A$かつ$\dip s_l+\frac{1}{2^l}\notin A$
  であるとすると, 
  $m=l+1$のとき, 
  (1) 
  $\dip s_l + \frac{1}{2^{l+1}} \in A$ の場合, 
  $a_{l+1}=1$であり, 
  $\dip s_{l+1}= s_l + \frac{1}{2^{l+1}}\in A$. (たった今そう場合分けした.)
  また, $\dip s_{l+1}+\frac{1}{2^{l+1}}=s_l + \frac{1}{2^{l+1}} + \frac{1}{2^{l+1}}
  = s_l + \frac{1}{2^l} \notin A$. ($m=l$のときの帰納法の仮定.)
  (2) 
  $s_l + \dip\frac{1}{2^{l+1}} \notin A$ の場合, 
  $a_{l+1}=0$であり, 
  $s_{l+1}= s_l \in A$. 
  また, $\dip s_{l+1}+\dip\frac{1}{2^{l+1}}=s_l + \frac{1}{2^{l+1}}\notin A$. 
  以上より, 
  どの番号$m=0,1,2,\dots$に対しても, 
  $s_m\in A$かつ$\dip s_m+\frac{1}{2^m}\notin A$が成り立つ. 


\end{proof}



\section{上限と下限\footnote{ここからは, 2021,Apr,17 以降の追加分.}}
$A\subset \R$ に対し, $A$の上界と下界を定める. 
実数$r$ が$A$の上界 (下界) であるとは, 任意の$A$の元$x\in A$に対し
$x \leqq r$ ($r \leqq x$)となることをいう. 



\begin{thebibliography}{20}
\par
  \bibitem{aomoto1} 青本和彦, 『微分と積分1』, 現代数学への入門, 岩波書店, 2003.
  \bibitem{saito1} 斎藤毅, 『集合・位相』, 東京大学出版会, 2009. 
  \bibitem{saito2} 斎藤毅, 『微積分』, 東京大学出版会, 2013.
  \bibitem{saito3} 斎藤毅, 「はじまりはコンパクト」, 『新・数学の学び方』, 岩波書店, 2015, pp.34-50. 
  \bibitem{sugi1} 杉浦光夫, 『解析入門$\mathrm{I}$』, 東京大学出版会, 1980. 
  \bibitem{1matsu1} 一松信, 『解析学序説 上』, 裳華房, 1962.
  \bibitem{1matsu2} 一松信, 『初等関数の数値計算』, シリーズ新しい応用の数学 8, 教育出版, 1974. 
  \bibitem{mori1} 森毅, 『現代の古典解析』, ちくま学芸文庫, 2006.
  \bibitem{mori2} 森毅, 『位相のこころ』, ちくま学芸文庫, 2006.
\end{thebibliography}
\end{document}