% -----------------------
% preamble
% -----------------------
% Don't change preamble code yourself. If you add something(usepackage, newtheorem, newcommand, renewcommand),
% please tell them editor of institutional paper of RUMS.

%documentclass
%------------------------
\documentclass[12pt, a4paper, dvipdfmx]{jsarticle}


%usepackage
%------------------------
\usepackage{amsmath}
\usepackage{amsthm}
%\usepackage[psamsfonts]{amssymb}
\usepackage{color}
\usepackage{ascmac}
\usepackage{amsfonts}
\usepackage{mathrsfs}
\usepackage{mathtools}
\usepackage{amssymb}
\usepackage{graphicx}
\usepackage{fancybox}
\usepackage{enumerate}
\usepackage{verbatim}
\usepackage{subfigure}
\usepackage{proof}
\usepackage{listings}
\usepackage{otf}
\usepackage{algorithm}
\usepackage{algorithmic}
\usepackage{tikz}
\usepackage[all]{xy}
\usepackage{amscd}

\usepackage{pb-diagram}

\usepackage[dvipdfmx]{hyperref}
\usepackage{xcolor}
\definecolor{darkgreen}{rgb}{0,0.45,0} 
\definecolor{darkred}{rgb}{0.75,0,0}
\definecolor{darkblue}{rgb}{0,0,0.6} 
\hypersetup{
    colorlinks=true,
    citecolor=darkgreen,
    linkcolor=darkred,
    urlcolor=darkblue,
}
\usepackage{pxjahyper}

\usepackage{enumitem}

\usepackage{bbm}

\usetikzlibrary{cd}

%theoremstyle
%--------------------------
\theoremstyle{definition}


%newtheoem
%--------------------------
%If you want to use theorem environment in Japanece, You can use these code. 
%Attention
%--------------------------
%all theorem enivironment number depend on only section number.
\newtheorem{Axiom}{公理}[section]
\newtheorem{Definition}[Axiom]{定義}
\newtheorem{Theorem}[Axiom]{定理}
\newtheorem{Proposition}[Axiom]{命題}
\newtheorem{Lemma}[Axiom]{補題}
\newtheorem{Corollary}[Axiom]{系}
\newtheorem{Example}[Axiom]{例}
\newtheorem{Claim}[Axiom]{主張}
\newtheorem{Property}[Axiom]{性質}
\newtheorem{Attention}[Axiom]{注意}
\newtheorem{Question}[Axiom]{問}
\newtheorem{Problem}[Axiom]{問題}
\newtheorem{Consideration}[Axiom]{考察}
\newtheorem{Alert}[Axiom]{警告}
\newtheorem{Notation}[Axiom]{記号}



%----------------------------
%If you want to use theorem environment with no number in Japanese, You can use these code.
\newtheorem*{Axiom*}{公理}
\newtheorem*{Definition*}{定義}
\newtheorem*{Theorem*}{定理}
\newtheorem*{Proposition*}{命題}
\newtheorem*{Lemma*}{補題}
\newtheorem*{Example*}{例}
\newtheorem*{Corollary*}{系}
\newtheorem*{Claim*}{主張}
\newtheorem*{Property*}{性質}
\newtheorem*{Attention*}{注意}
\newtheorem*{Question*}{問}
\newtheorem*{Problem*}{問題}
\newtheorem*{Consideration*}{考察}
\newtheorem*{Alert*}{警告}

%--------------------------
%If you want to use theorem environment in English, You can use these code.
%--------------------------
%all theorem enivironment number depend on only section number.
\newtheorem{Axiom+}{Axiom}[section]
\newtheorem{Definition+}[Axiom+]{Definition}
\newtheorem{Theorem+}[Axiom+]{Theorem}
\newtheorem{Proposition+}[Axiom+]{Proposition}
\newtheorem{Lemma+}[Axiom+]{Lemma}
\newtheorem{Example+}[Axiom+]{Example}
\newtheorem{Corollary+}[Axiom+]{Corollary}
\newtheorem{Claim+}[Axiom+]{Claim}
\newtheorem{Property+}[Axiom+]{Property}
\newtheorem{Attention+}[Axiom+]{Attention}
\newtheorem{Question+}[Axiom+]{Question}
\newtheorem{Problem+}[Axiom+]{Problem}
\newtheorem{Consideration+}[Axiom+]{Consideration}
\newtheorem{Alert+}{Alert}

%commmand
%----------------------------
\newcommand{\N}{\mathbb{N}}
\newcommand{\Z}{\mathbb{Z}}
\newcommand{\Q}{\mathbb{Q}}
\newcommand{\R}{\mathbb{R}}
\newcommand{\C}{\mathbb{C}}
\newcommand{\F}{\mathcal{F}}
\newcommand{\X}{\mathcal{X}}
\newcommand{\Y}{\mathcal{Y}}
\newcommand{\Hil}{\mathcal{H}}
\newcommand{\RKHS}{\Hil_{k}}
\newcommand{\Loss}{\mathcal{L}_{D}}
\newcommand{\MLsp}{(\X, \Y, D, \Hil, \Loss)}
\newcommand{\p}{\partial}
\newcommand{\h}{\mathscr}
\newcommand{\mcal}{\mathcal}
\newcommand{\lan}{\langle}
\newcommand{\ran}{\rangle}
\newcommand{\pal}{\parallel}
\newcommand{\dip}{\displaystyle}
\newcommand{\e}{\varepsilon}
\newcommand{\dl}{\delta}
\newcommand{\pphi}{\varphi}
\newcommand{\ti}{\tilde}

\renewcommand{\P}{\mathbb{P}}
\newcommand{\Probsp}{(\Omega, \F, \P)}

%new definition macro
%-------------------------
\def\inner<#1>{\langle #1 \rangle}

\usepackage[stable]{footmisc}
\renewcommand{\proofname}{\bf 証明} % 「証明」の見出しを日本語にする
\newcommand{\pr}{\mathop{\mathrm{pr}}\nolimits} % 射影の記号を斜字体にしない
\newcommand{\id}{\mathop{\mathrm{id}}\nolimits} % 恒等写像
\newcommand{\Ob}{\mathop{\mathrm{Ob}}\nolimits}
\numberwithin{equation}{section} % 式番号を「(3.5)」のように印刷
\newcommand{\Hom}{\mathop{\mathrm{Hom}}\nolimits}
\newcommand{\Mod}{\mathop{\mathrm{Mod}}\nolimits}
\newcommand{\End}{\mathop{\mathrm{End}}\nolimits}
\newcommand{\Aut}{\mathop{\mathrm{Aut}}\nolimits}
\newcommand{\Mor}{\mathop{\mathrm{Mor}}\nolimits}

% 圏の記号など
\newcommand{\Set}{{\bf Set}}
\newcommand{\Vect}{{\bf Vect}}
\newcommand{\FDVect}{{\bf FDVect}}
\newcommand{\Ring}{{\bf Ring}}
\newcommand{\Ab}{{\bf Ab}}
\newcommand{\CGA}{{\bf CGA}}
\newcommand{\GVect}{{\bf GVect}}
\newcommand{\Lie}{{\bf Lie}}
\newcommand{\dLie}{{\bf Liec}}


%\usepackage{eufrak}
\def\rnum#1{\expandafter{\romannumeral #1}} 
\def\Rnum#1{\uppercase\expandafter{\romannumeral #1}} 

\newcommand{\bm}[1]{{\mbox{\boldmath $#1$}}}

\newcommand{\gG}{\mathfrak{g}}
\newcommand{\gh}{\mathfrak{h}}
\newcommand{\lL}{\mathop{\mathrm{L}}\nolimits}

\newcommand{\cat}[1]{\textup{\textsf{#1}}}% for categories
\newcommand{\fun}[1]{\textup{#1}}%for functors

%for yoneda
\font\maljapanese=dmjhira at 2ex % you can change this "2ex" value
\def\yo{\textrm{\maljapanese\char"48}}


%----------------------------
%documenet 
%----------------------------
% Your main code must be written between begin document and end document.
\title{Lie環と代数の随伴}
\author{Toshi2019}
\date{}
\begin{document}

\maketitle

Lie 環と代数の間の随伴についてまとめたノート. 
\[
\begin{tikzcd}[column sep=large]
\cat{Lie}_k 
\arrow[r, bend left, "U"] 
\arrow[r, phantom, "\perp"]  
&
\cat{Alg}_k 
\arrow[l, bend left, "\lL"]
\end{tikzcd}\label{adj-}
\]

% 随伴テンプレート
\begin{comment}
\begin{equation}
    \begin{tikzcd}
    \cat{C}
    \arrow[r, bend left, "F"] 
    \arrow[r, phantom, "\perp"] 
    & 
    \cat{C}' \arrow[l, bend left, "G"]
    \end{tikzcd}
\end{equation}
\end{comment}


\section{主定理}

$k$ を単位元を持つ可換環とする. 
$k$ 代数と$k$ 代数射のなす圏を$\cat{Alg}_k$, 
$k$ 上の Lie 環と Lie 環射のなす圏を$\cat{Lie}_k$
とかく. 
次の \eqref{alg-lie} を示すのが本稿の目的である. 
\begin{align}
    \Hom_{\cat{Alg}_k}(U(\gG),A)
    \cong
    \Hom_{\cat{Lie}_k}(\gG,\lL(A)) \label{alg-lie}
\end{align}
ここで, 
$U: \cat{Lie}_k\to\cat{Alg}_k$ 
は Lie 環 $\gG$ に対し, 
包絡環 $U(\gG)$ を対応させ, 
Lie 環射 $f: \gG \to \gG'$に対し, 
代数射 $U(f): U(\gG) \to U(\gG')$ 
を対応させる関手である. 
また, 
$L: \cat{Alg}_k \to \cat{Lie}_k$
は, 代数 $A$ に対し Lie 環を, 
代数射 $h: A\to A'$ に対し 
Lie 環射を, 後述する方法\eqref{bracket}で対応させる関手である. 

\section{Lie 環}

Lie 環 $(\gG, [\bullet,\bullet])$とは, 
$k$ 線形空間 $\gG$ と $k$ 双線形写像
$[\bullet,\bullet]: \gG\times\gG\to\gG$
の組で次の条件を満たすものをいう. 

\begin{quote}
    (L1) 任意の $x\in \gG$ に対し $[x,x] = 0$\\
    (L2) 任意の $x,y,z\in \gG$ に対し 
    $[x,[y,z]] + [y,[z,x]] + [z,[x,y]] = 0$
\end{quote}



\section{代数}

$k$ を単位元を持つ可換環とする. 
$k$ 代数 $(A, m, u)$ とは, $k$ 線形空間 $A$ と
$k$ 線形写像$m: A\otimes A\to A$, 
$u: k \to A$
の三つ組で次の図式を可換にするものをいう. 
\[
\begin{tikzcd} 
A \otimes A \otimes A
\arrow[r, "{m\otimes \id}"] 
\arrow["{\id \otimes m}"',d] 
& 
A \otimes A 
\arrow[d, "m"] 
& & 
k \otimes A 
\arrow[r, "u\otimes \id"] 
\arrow[dr, "\sim" sloped] 
& 
A \otimes A 
\arrow[d, "m"] 
& 
A \otimes k
\arrow[l, "\id \otimes u"']
\arrow[dl, "\sim"' sloped]\\ 
A \otimes A 
\arrow[r, "m"] 
%\arrow[ur, dashed, "i"] 
& 
A 
& & &
A 
\end{tikzcd}
\]

以下, $k$ を取り替えないので, 
$k$ 代数をたんに代数とよぶ. 
$(A,m_A,u_A), (B,m_B,u_B)$ を代数とする. 
線形写像 $f:A\to B$ が代数射であるとは, 
次の図式が可換となることをいう. 
\[
    \begin{tikzcd}
        A\otimes A
        \arrow[r, "m_A"]
        \arrow[d, "f\otimes f"']
        & 
        A
        \arrow[d, "f"]
        & & &%=============================
        k
        \arrow[r, "u_A"]\arrow[d, equals]
        &
        A
        \arrow[d, "f"]\\%===============
        B\otimes B
        \arrow[r,"m_B"]
        &
        B
        & & &%=============================
        k
        \arrow[r,"u_B"]
        &
        B
    \end{tikzcd}
\]

代数$A$に対し, 次の方法でブラケット積を定める. 
\begin{align}
    [x,y] = xy - yx \label{bracket}    
\end{align}
これにより, $A$ から Lie環$\lL(A)$が定まる. 
このとき, 代数射 $f: A \to B$
に対し, 
\begin{align*}
    f([x,y]) 
    &= f([x,y]) = f(xy) - f(yx)\\
    &= f(x)f(y) - f(y)f(x)
    = [f(x), f(y)]
\end{align*}
が成り立つので, $f$ を自然にLie環射とみなせる. 
これを$L(f):\lL(A) \to \lL(B)$とかく. 

\section{包絡環}
$\gG$ をLie環とする. 

\section{証明}



\begin{thebibliography}{20}
\par
  \bibitem[阿部 77]{Abe} 阿部英一, ホップ代数, 岩波書店, 1977.
  \bibitem[谷崎 02]{Tani} 谷崎俊之, リー代数と量子群, 共立叢書 現代数学の潮流, 共立出版, 2002.
  \bibitem[Kas95]{Kas} C.\ Kassel, Quantum Groups, 
  Graduate Texts in Mathematics, 
  vol.\ 155, Springer-Verlag, New York, 1995.
\end{thebibliography}
\end{document}